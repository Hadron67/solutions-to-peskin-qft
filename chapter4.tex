\section{Chapter 4}

\setcounter{equation}{137}
\problem Let us return to the problem of the creation of Klein-Gordon particles
by a classical source. Recall from Chapter 2 that this process can be described
by the Hamitonian
\begin{equation*}
    H = H_0 + \int \dd^3 x \, \left(-j(t, \vec{x})\phi(x)\right),
\end{equation*}
where $H_0$ is the free Klein-Gordon Hamitonian, $\phi(x)$ is the Klein-Gordon 
field, and $j(x)$ is a c-number scalar function. We found that, if the system is
in the vacuum state before the source is turned on, the source will create a mean
number of particles
\begin{equation*}
    \avg{N} = \int \frac{\dd^3 p}{(2\pi)^3} \frac{1}{2E_{\vec{p}}} \abs{\tilde{\jmath}(p)}^2.
\end{equation*}
In this problem we will verify that statement, and extract more detailed information,
by using a perturbation expansion in the strength of the source.
\begin{problembody}
    \item Show that the probability that the source creates \textit{no} particles is 
    given by 
    \begin{equation*}
        P(0) = \abs{
            \bra{0} T\left\{
                \exp[
                    i\int \dd^4x \, j(x) \phi_I(x)
                ]    
            \right\}\ket{0}
        }^2
    \end{equation*}

    \item\label{ex:cp4:eval_p} Evaluate the term in $P(0)$ of order $j^2$, and show that $P(0) = 1 - \lambda + \bigo(j^4)$,
    where $\lambda$ equals the expression given above for $\avg{N}$.

    \item\label{ex:cp4:p_sum} Represent the term computed in part \ref{ex:cp4:eval_p} as a Feynman diagram. Now Represent
    the whole perturbation series for $P(0)$ in terms of Feynman diagrams. Show that this series
    exponentiates, so that it can be summed exactly: $P(0) = \exp(-\lambda)$.

    \item Compute the probability that the source creates one particle of momentum $k$.
    Perform this computation first to $\bigo(j)$ and then to all orders, using the trick of part \ref{ex:cp4:p_sum} to sum
    the series.

    \item Show that the probability of producing $n$ particles is given by 
    \begin{equation*}
        P(n) = (1 / n!)\lambda^n \exp(-\lambda).
    \end{equation*}
    This is a \textit{Poisson distribution}.

    \item Prove the following facts about the Poisson distribution:
    \begin{equation*}
        \sum_{n = 0}^\infty P(n) = 1; 
        \qquad \avg{N} = \sum_{n = 0}^\infty n P(n) = \lambda.
    \end{equation*}
    The first identity says that the $P(n)$'s are properly normalized probabilities, while the second confirms our proposal
    for $\avg{N}$. Compute the mean square fluctuation $\avg{(N - \avg{N})^2}$.
\end{problembody}

\solution
\begin{problembody}
    \item Work in interaction picture, where all field operators evaluate according to $H_0$, and all states to the full 
    Hamitonian. The evaluation operator of states is
    \begin{equation*}
        U_I(t) = e^{iH_0t} e^{-iHt},
    \end{equation*}
    whose time derivative is
    \begin{equation*}
        i \frac{\dd}{\dd t} U_I(t)
        = e^{iH_0t}(H - H_0)e^{-iH_0t} e^{iH_0t} e^{-iHt}
        = H_I(t) U_I(t).
    \end{equation*}
    Thus $U_I(t)$ can also be written as
    \begin{equation*}
        U_I(t) = T e^{-i \int_0^t H_I(t) \dd t} 
        = T \exp[i\int_0^t \dd t\int \dd^3x \, j(x)\phi_I(x)].
    \end{equation*}
    Assume the source is turned on during the interval $[T, -T]$, the probability of vacuum to vacuum is given by
    \begin{equation*}
        P(0) = \abs{\inner{0, T}{0, -T}}^2 = \abs{\bra{0} U_I(T) U_I(-T) \ket{0}}^2.
    \end{equation*}
    Take the limit $T \to +\infty$, it becomes
    \begin{equation*}
        P(0) = \abs{
            \bra{0} T\left\{
                \exp[
                    i\int \dd^4x \, j(x) \phi_I(x)
                ]    
            \right\}\ket{0}
        }^2 \equiv \abs{M}^2.
    \end{equation*}

    \item To do this, expand $M$ to order $j^4$:
    \begin{equation*}
        M = \bra{0} T\left[1 
        + i\int \dd^4x \, j(x)\phi_I(x) 
        - \frac{1}{2} \int \dd^4x \int \dd^4y \, j(x)j(y)\phi_I(x)\phi_I(y)
        + F
        + \bigo(j^4) \right]\ket{0}.
    \end{equation*}
    Where $F$ is an integral containing three $\phi_I$'s. By wick's theorem, the second and fourth term vanishes, 
    the two $\phi_I$'s in the third term becomes a propagator:
    \begin{equation*}
        M = 1 - \frac{1}{2} \int \dd^4x \int \dd^4y \, j(x)D_F(x - y)j(y) + \bigo(j^4).
    \end{equation*}
    The integral can be evaluated as below:
    \begin{align*}
        \int \dd^4x \int \dd^4y \, j(x)D_F(x - y)j(y)
        & = \int \dd^4x \int \dd^4y \int \frac{\dd^4p}{(2\pi)^4}
        j(x)j(y)\frac{ie^{-ip\cdot(x - y)}}{p^2 - m^2 + i\epsilon}\\
        %
        & = \int \frac{\dd^4p}{(2\pi)^4} \frac{i\abs{\tilde{\jmath}(p)}^2}{p^2 - m^2 + i\epsilon}\\
        & = \int \frac{\dd^3p}{(2\pi)^3} \frac{\abs{\tilde{\jmath}(p)}^2}{2E_{\vec{p}}}\\
        & = \lambda.
    \end{align*}
    Thus $P(0) = \abs{M}^2 = 1 - \lambda + \bigo(j^4)$.

    \item The (position space) Feynman rule for the total Hamitonian is
    \begin{equation*}
        \begin{tikzpicture}
            \node [vertex, label = above:$x$] (x) at (0, 0) {};
            \node [vertex, label = above:$y$] (y) at (1, 0) {};
            \draw (0, 0) -- (1, 0);
        \end{tikzpicture}
        = D_F(x - y);
        \qquad \begin{tikzpicture}
            \node [cross vertex, label = above:$x$] (x) at (0, 0) {};
            \draw (0, 0) -- (1, 0);
        \end{tikzpicture}
        = ij(x),
    \end{equation*}
    and integrating over all internal points. Then $\lambda$ can be represented as
    \begin{equation*}
        \lambda = \int \dd^4x \int \dd^4y \, j(x)D_F(x - y)j(y) = 
        -\begin{tikzpicture}
            \node [cross vertex] (x) at (0, 0) {};
            \node [cross vertex] (y) at (1, 0) {};
            \draw (0, 0) -- (1, 0);
        \end{tikzpicture}.
    \end{equation*}
    As we can see, $M$ is the sum of all diagrams without external vertices. Since only diagrams with an even
    number of vertices can be fully connected to yield a non-zero term, we now consider $2n$-th diagram in the 
    perturbation series. It is given by
    \begin{equation*}
        \frac{1}{(2n)!} S \left(
            \begin{tikzpicture}
                \node [cross vertex] (x) at (0, 0) {};
                \node [cross vertex] (y) at (1, 0) {};
                \draw (0, 0) -- (1, 0);
            \end{tikzpicture}    
        \right)^n
        = \frac{(-\lambda)^n}{(2n)!} S,
    \end{equation*}
    where $S$ is the number of ways these vertices could be connected. We'll find it in this way: first choose 2 
    vertices from $2n$ vertices to connect, then choose 2 more from the rest $2n - 2$ vertices, and so on, and 
    divide the final result by $n!$ to cancel the order of our steps. Thus
    \begin{equation*}
        S = \frac{1}{n!} C_{2n}^2 C_{2n - 2}^2 \cdots C_2^2
        = \frac{1}{n!} \frac{(2n)!}{2(2n - 2)!} \frac{(2n - 2)!}{2(2n - 4)!}
        \cdots \frac{2!}{2(0!)}
        = \frac{(2n)!}{2^n n!}.
    \end{equation*}
    Substitute into the expression of $2n$-th term, the $(2n)!$ cancels, resulting in another taylor series
    of exponentiate:
    \begin{equation*}
        M = \frac{1}{n!}(-\frac{\lambda}{2})^n = e^{- \lambda / 2}.
    \end{equation*}
    So that $P(0) = \abs{M}^2 = e^{-\lambda}$.

    \item In this case, the initial and final state are
    \begin{equation*}
        \ket{i} = \ket{0}, \qquad \text{and} \quad \ket{f} = a_k^\dagger\ket{0} = \frac{1}{\sqrt{2\omega_k}}\ket{k},
    \end{equation*}
    respectively. The probability is now
    \begin{equation*}
        P(1, k) \equiv \abs{M_k}^2 = \abs{
            \bra{f} T\left\{
                \exp[i\int \dd^4x \, j(x)\phi_I(x)]    
            \right\} \ket{0}
        }^2.
    \end{equation*}
    When expanding, we need the contraction of $\phi_I$ with $\ket{f}$:
    \begin{equation*}
        \wick{\c \phi_I(x) |\c f\rangle}
        = \wick{\c \phi_I(x) \c a_k^\dagger}\ket{0}
        = \sum_{k'} \varphi_{k'} a_{k'} a_k^\dagger \ket{0}
        = \varphi_k \ket{0}
        = \frac{1}{\sqrt{2V\omega_k}} e^{-ik\cdot x}\ket{0}.
    \end{equation*}
    This yields the following rule for external lines:
    \begin{align*}
        \begin{tikzpicture}[scale = 0.5]
            \node [cross vertex] at (0, 0) {};
            \node [label = above:$k$] at (1, 0.5) {};
            \draw (0, 0) -- (2, 0);
            \draw [-Latex] (1.5, 0.5) -- (0.5, 0.5);
        \end{tikzpicture}
        = i\int \dd^4 x \, j(x) \wick{\c\phi_I(x) \c a_k^\dagger}
        = \frac{i\tilde{\jmath}(k)}{\sqrt{2V\omega_k}}
        \equiv i\mu_k.
    \end{align*}
    Since the final state must contract with one field operator to form a vacuum state, every diagram should contain 
    one vertex that connected to an external line. Thus only $(2n + 1)$-th order 
    diagrams are non-zero, for $n \geqslant 0$. They are given by
    \begin{equation*}
        \frac{1}{(2n + 1)!} S_1 \Big(\begin{tikzpicture}[scale = 0.5]
            \node [cross vertex] at (0, 0) {};
            \node [label = above:$k$] at (1, 0.5) {};
            \draw (0, 0) -- (2, 0);
            \draw [-Latex] (0.5, 0.5) -- (1.5, 0.5);
        \end{tikzpicture}\Big) \left(\begin{tikzpicture}
            \node [cross vertex] at (0, 0) {};
            \node [cross vertex] at (1, 0) {};
            \draw (0, 0) -- (1, 0);
        \end{tikzpicture}\right)^n
        = \frac{i}{(2n + 1)!} S_1 \mu_k^\ast (-\lambda)^n,
    \end{equation*}
    where $S_1$, again, is the number of ways these vertices can be connected, can be calculated as
    \begin{equation*}
        S_1 = A_{2n + 1}^1 S = \frac{(2n + 1)(2n)!}{2^nn!}. 
    \end{equation*}
    The diagram then becomes
    \begin{equation*}
        \frac{i}{n!} \left(-\frac{\lambda}{2}\right)^n \mu_k^\ast.
    \end{equation*}
    Thus
    \begin{equation*}
        P(1, k) = \abs{
            i\mu_k^\ast \sum_{n = 0}^\infty
            \frac{1}{n!}
            \left(-\frac{\lambda}{2}\right)^n
        }^2 = \abs{\mu_k}^2 e^{-\lambda}
        = \frac{\abs{\tilde{\jmath}(k)}^2}{2V\omega_k} e^{-\lambda}.
    \end{equation*}

    \item Consider first, the probability of creating $n$ particles with momenta $k_1, k_2, \cdots, k_n$, respectively. The final state is
    \begin{equation*}
        \ket{f} = \prod_{i = 1}^n a_{k_i}^\dagger \ket{0},
    \end{equation*}
    Now $M$ should be the sum of all diagrams with $n$ vertices connected to $n$ different external lines, so only $(2j + n)$-th order diagrams are non-zero:
    \begin{equation*}
        \frac{1}{(2j + n)!} S_n \Big(\prod_{l = 1}^n \begin{tikzpicture}[scale = 0.5]
            \node [cross vertex] at (0, 0) {};
            \node [label = above:$k_l$] at (1, 0.5) {};
            \draw (0, 0) -- (2, 0);
            \draw [-Latex] (0.5, 0.5) -- (1.5, 0.5);
        \end{tikzpicture}\Big) \left(\begin{tikzpicture}
            \node [cross vertex] at (0, 0) {};
            \node [cross vertex] at (1, 0) {};
            \draw (0, 0) -- (1, 0);
        \end{tikzpicture}\right)^j
        = \frac{1}{(2j + n)!} S_n \left(\prod_{l = 1}^n i\mu_{k_l}^\ast\right) (-\lambda)^j.
    \end{equation*}
    Where $S_n$ is the number of contraction ways:
    \begin{equation*}
        S_n = A_{2j + n}^n S 
        = \frac{(2j + n)!}{(2j)!} \frac{(2j)!}{2^jj!} 
        = \frac{(2j + n)!}{2^j j!}.
    \end{equation*}
    Thus
    \begin{equation*}
        M = 
        \left(\prod_{l = 1}^n i\mu_{k_l}^\ast\right) 
        \sum_{j = 0}^\infty 
        j! \left(-\frac{\lambda}{2}\right)^j
        = \left(\prod_{l = 1}^n i\mu_{k_l}^\ast\right)
        e^{-\lambda / 2}.
    \end{equation*}
    The probability of creating $n$ particles is then $\abs{M}^2$ sum over the $n$ momenta, and divide by $n!$
    to cancel the overcounting when interchange these momenta:
    \begin{align*}
        P(n) = \frac{1}{n!}\sum_{\{k_i\}} \abs{M}^2 
        & = \left(
            \sum_k
            \frac{\abs{\tilde{\jmath}(k)}^2}{2V\omega_k}
        \right)^n
        \frac{1}{n!} e^{-\lambda}\\
        & = \left(
            \int \frac{\dd^3k}{(2\pi)^3} \frac{\abs{\tilde{\jmath}(k)}^2}{2\omega_k}    
        \right)^n \frac{1}{n!} e^{-\lambda}\\
        & = \frac{\lambda^n}{n!} e^{-\lambda}.
    \end{align*}

    \item \begin{equation*}
        \sum_{n = 0}^\infty P(n) 
        = e^{-\lambda} \sum_{n = 0}^\infty \frac{\lambda^n}{n!}
        = e^{-\lambda} e^{\lambda} = 1.
    \end{equation*}
    \begin{equation*}
        \avg{N} = \sum_{n = 0}^\infty nP(n)
        = e^{-\lambda} \sum_{n = 1}^\infty \frac{\lambda^n}{(n - 1)!}
        = \lambda e^{-\lambda} \sum_{n = 0}^\infty \frac{\lambda^n}{n!}
        = \lambda.
    \end{equation*}
    \begin{align*}
        \avg{(N - \avg{N})^2} = \avg{N^2} - \avg{N}^2
        & = \avg{N(N - 1)} + \avg{N} - \avg{N}^2\\
        & = e^{-\lambda} \sum_{n = 2}^\infty \frac{\lambda^n}{(n - 2)!}
        + \lambda - \lambda^2\\
        & = \lambda.
    \end{align*}
\end{problembody}

\problem \textbf{Decay of a scalar particle.} Consider the following Lagrangian, involving two
real scalar fiels $\Phi$ and $\phi$:
\begin{equation*}
    \lag = \frac{1}{2}(\partial_\mu\Phi)^2 - \frac{1}{2}M^2\Phi^2
    + \frac{1}{2}(\partial_\mu\phi)^2 - \frac{1}{2}m^2\phi^2
    - \mu\Phi\phi\phi.
\end{equation*}
The last term is an interaction that allows a $\Phi$ particle to decay into two $\phi$'s, provided
that $M > 2m$. Assume that this condition is met, calculate the lifetime of the $\Phi$ to lowest order
in $\mu$.

\solution 
Feynman rule for this Lagrangian is
\begin{equation*}
    \begin{tikzpicture}[baseline = (o)]
        \node [label = above:$k$] (o) at (0, 0) {};
        \draw (-1, 0) -- (1, 0) [arrowed];
    \end{tikzpicture} = \frac{i}{k^2 - m^2 + i\epsilon},
    \qquad \begin{tikzpicture}[baseline = (o)]
        \node [label = above:$k$] (o) at (0, 0) {};
        \draw [double line] (-1, 0) -- (1, 0) [arrowed];
    \end{tikzpicture} = \frac{i}{k^2 - M^2 + i\epsilon},
\end{equation*}
\begin{equation*}
    \begin{tikzpicture}[baseline = (o.south)]
        \coordinate (a) at (-1, 0);
        \coordinate (b) at (60:1);
        \coordinate (c) at (-60:1);
        \draw [double line] (a) -- (0, 0);
        \draw (b) -- (0, 0) -- (c);
        \node [vertex] (o) at (0, 0) {};
    \end{tikzpicture} = i\mu.
\end{equation*}
The lowest order contribution to the decay of $\Phi$ is given by the tree diagram:
\begin{center}
    \begin{tikzpicture}[scale = 1.5]
        \coordinate (a) at (0, -1);
        \coordinate (b) at (30:1);
        \coordinate (c) at (150:1);
        \draw [double line] (a) -- (0, 0)[arrowed, text = $k$];
        \draw (0, 0) -- (b) [arrowed, text = $k_2$];
        \draw (0, 0) -- (c) [arrowed, text = $k_1$];
        \node [vertex] at (0, 0) {};
    \end{tikzpicture}.
\end{center}
Thus $i\ims = i\mu$, and the decay width
\begin{align*}
    \dd \Gamma & = \frac{1}{2M(2\pi)^2} 
    \int \dd^3k_1 \int \dd^3k_2 \, \frac{\abs{\ims}^2}{4E_1E_2} \delta(E_1 + E_2 - 2M)\delta^3(k_1 + k_2)\\
    & = \frac{1}{64\pi^2M} \int \dd\Omega 
    \int_0^{+\infty} k^2\dd k \frac{\abs{\ims}^2}{k^2 + m^2}\delta(\sqrt{k^2 + m^2} - M)\\
    %
    & = \frac{\mu^2}{64\pi^2M} \sqrt{1 - \frac{m^2}{M^2}}.
\end{align*}

\problem \textbf{Linear sigma model.} The interactions of pions at low energy can be described
by a phenomenological model called the \textit{linear sigma model}. Essentially, this model 
consists of $N$ real scalar fields coupled by a $\phi^4$ interaction that is symmetric under 
rotations of the $N$ fields. More specifically, let $\Phi^i(x)$, $i = 1, \cdots, N$ be a set of 
$N$ fields, governed by the Hamitonian
\begin{equation*}
    H = \int \dd^3x \, \left(
        \frac{1}{2}(\Pi^i)^2 + \frac{1}{2}(\nabla\Phi^i)^2 + V(\Phi^2)    
    \right),
\end{equation*}
where $(\Phi^i)^2 = \Phi\cdot\Phi$, and
\begin{equation*}
    V(\Phi^2) = \frac{1}{2}m^2(\Phi^i)^2 
    + \frac{\lambda}{4}\left((\Phi^i)^2\right)^2
\end{equation*}
is a function symmetric under rotations of $\Phi$. For (classical) field configurations of $\Phi^i(x)$
that are constant in space and time, this term gives the only contribution to $H$; hence, $V$ is the field
potential energy.

(What does this Hamitonian have to do with the strong interactions? There are two types of light quarks, 
$u$ and $d$. These quarks have identical strong interactions, but different masses. If these quarks are
massless. If these quarks are massless, the Hamitonian of the strong interactions is invariant to unitary
transformations of the 2-component object $(u, d)$:
\begin{equation*}
    \begin{pmatrix}
        u \\ d
    \end{pmatrix} \to \exp(i\vec{\alpha}\cdot\vec{\sigma} / 2)
    \begin{pmatrix}
        u \\ d
    \end{pmatrix}.
\end{equation*}
This is transformation is called an \textit{isospin} rotation. If, in addition, the strong interactions 
are described by a vector ``gluon'' field (as is true in QCD), the strong interaction Hamitonian is invariant
to the isospin rotations done separately on the left-handed and right handed components of the quark fields.
Thus, the complete symmetry of QCD with two massless quarks is $SU(2)\times SU(2)$. It happens that $SO(4)$,
the group of rotations in 4 dimensions, is isomorphic to $SU(2)\times SU(2)$
\footnote{I wonder if this is a mistake, or expressed wrongly, since $SO(4)$ is not a direct product of Lie groups.}, 
so for $N = 4$, the linear 
sigma model has the same symmetry group as the strong interactions.)
\begin{problembody}
    \item Analyze the linear sigma model for $m^2 > 0$ by noticing that, for $\lambda = 0$, the Hamitonian
    given above is exactly $N$ copies of the Klein-Gordon Hamitonian. We can the calculate scattering amplitudes
    as perturbation series in the parameter $\lambda$. Show that the propagator is
    \begin{equation*}
        \wick{\c \Phi^i(x) \c \Phi^j(y)} = \delta^{ij}D_F(x - y),
    \end{equation*}
    where $D_F$ is the standard Klein-Gordon propagator for mass $m$, and that there is one type of vertex gien 
    by
    \begin{equation*}
        \begin{tikzpicture}[baseline = (o.south)]
            \node (k) at (-1,  1) {$k$};
            \node (l) at ( 1,  1) {$l$};
            \node (i) at (-1, -1) {$i$};
            \node (j) at ( 1, -1) {$j$};
            \draw (k) -- (0, 0) -- (l);
            \draw (i) -- (0, 0) -- (j);
            \node (o) [vertex] at (0, 0) {};
        \end{tikzpicture}
        = -2i\lambda(
            \delta^{ij}\delta^{kl}
            + \delta^{il}\delta^{jk}
            + \delta^{ik}\delta^{jl}
        ).
    \end{equation*}
    (That is, the vertex between two $\Phi^1$s and two $\Phi^2$s has the value $(-2i\lambda)$; that between four 
    $\Phi^1$s has the value $(-6i\lambda)$.) Compute, to leading order in $\lambda$, the differential cross sections 
    $\dd\sigma / \dd\Omega$, in the center of mass frame, for the scattering processes
    \begin{equation*}
        \Phi^1\Phi^2 \to \Phi^1\Phi^2, 
        \qquad \Phi^1\Phi^1 \to \Phi^2\Phi^2,
        \qquad \text{and} \qquad
        \Phi^1\Phi^1 \to \Phi^1\Phi^1
    \end{equation*}
    as functions of the center-of-mass energy.

    \item Now consider the case $m^2 < 0$: $m^2 = -\mu^2$. In this case, $V$ has a local maximum, rather than a minimum,
    at $\Phi^i = 0$. Since $V$ is a potential energy, this implies that the ground state of the theory is not near $\Phi^i = 0$
    but rather is obtained by shifting $\Phi^i$ toward the minimum of $V$. By rotational invariance, we can consider this 
    shift to be in the $N$th direction. Write, then,
    \begin{align*}
        \Phi^i(x) & = \pi^i(x), \qquad i = 1, \cdots, N - 1,\\
        \Phi^N(x) & = v + \sigma(x),
    \end{align*}
    where $v$ is a constant chosen to minimize $V$. (The notation $\pi^i$ suggests a poin field and should not be confused
    with a canonical momentum.) Show that, in these new coordinates (and substituting for $v$ its expression in terms of $\lambda$
    and $\mu$), we have a theory of a massive $\sigma$ field and $N - 1$ \textit{massless} pion fields, interacting through 
    cubic and quartic potential energy terms which all become small as $\lambda \to 0$. Construct the Feynman rules by assigning
    values to the propagators and vertices:
    \begin{align*}
        \wick{\c \sigma \c \sigma} & = \qquad
        \begin{tikzpicture}[baseline = (o.south)]
            \node (o) [empty]  at (0, 0) {};
            \draw [double line] (0, 0) -- (1.5, 0) [arrowed];
        \end{tikzpicture}
        \qquad\qquad
        \begin{tikzpicture}[baseline = (o.south), scale = 0.6]
            \node (i) at (-150:1.3) {$i$};
            \node (j) at ( -30:1.3) {$j$};
            \draw (-150:1) -- (0, 0) -- (-30:1);
            \draw [double line] (0, 1) -- (0, 0);
            \node (o) [vertex] at (0, 0) {};
        \end{tikzpicture}
        \qquad
        \begin{tikzpicture}[baseline = (o.south), scale = 0.6]
            \node (k) [empty] at (0, 1) {};
            \node (i) [empty] at (-150:1) {};
            \node (j) [empty] at (-30:1) {};
            \draw [double line] (i) -- (0, 0) -- (j);
            \draw [double line] (k) -- (0, 0);
            \node (o) [vertex] at (0, 0) {};
        \end{tikzpicture}\\
        %
        \wick{\c \pi^i \c \pi^j} & = \qquad
        \begin{tikzpicture}[baseline = (i.south)]
            \node (j) [empty, label = right:$j$] at (1.5, 0) {};
            \node (i) [empty, label = left:$i$] at (0, 0) {};
            \draw (0, 0) -- (1.5, 0) [arrowed];
        \end{tikzpicture}
        \qquad\qquad
        \begin{tikzpicture}[baseline = (o.south), scale = 0.6]
            \node (o) [vertex] at (0, 0) {};
            \draw (135:1) -- (0, 0) -- (45:1);
            \draw (-135:1) -- (0, 0) -- (-45:1);
            \node at (135:1.2) {$k$};
            \node at (45:1.2) {$l$};
            \node at (-135:1.2) {$i$};
            \node at (-45:1.2) {$j$};
        \end{tikzpicture}
        \qquad
        \begin{tikzpicture}[baseline = (o.south), scale = 0.6]
            \draw [double line] (135:1) -- (0, 0) -- (45:1);
            \node (o) [vertex] at (0, 0) {};
            \draw (-135:1) -- (0, 0) -- (-45:1);
            \node at (-135:1.2) {$i$};
            \node at (-45:1.2) {$j$};
        \end{tikzpicture}
        \qquad
        \begin{tikzpicture}[baseline = (o.south), scale = 0.6]
            \draw [double line] (135:1) -- (0, 0) -- (45:1);
            \draw [double line] (-135:1) -- (0, 0) -- (-45:1);
            \node (o) [vertex] at (0, 0) {};
        \end{tikzpicture} 
    \end{align*}

    \item Compute the scattering amplitude for the process
    \begin{equation*}
        \pi^i(p_1)\pi^j(p_2) \to \pi^k(p_3)\pi^l(p_4)
    \end{equation*}
    to leading order in $\lambda$. There are now four Feynman diagrams that contribute:
    \begin{equation*}
        \begin{tikzpicture}[baseline = (o.south), scale = 0.8]
            \coordinate (a) at (0, 0.6);
            \coordinate (b) at (0, -0.6);
            \node (o) [empty] at (0, 0) {};
            \draw [double line] (b) -- (a);
            \draw (a) + (150:1) -- (a) -- +(30:1);
            \draw (b) + (-150:1) -- (b) -- +(-30:1);
        \end{tikzpicture}
        \qquad + \qquad
        \begin{tikzpicture}[baseline = (o.south), scale = 0.8]
            \coordinate (a) at (0.6, 0);
            \coordinate (b) at (-0.6, 0);
            \node (o) [empty] at (0, 0) {};
            \draw [double line] (b) -- (a);
            \draw (a) + (60:1) -- (a) -- +(-60:1);
            \draw (b) + (120:1) -- (b) -- +(-120:1);
        \end{tikzpicture}
        \qquad + \qquad
        \begin{tikzpicture}[baseline = (o.south), scale = 0.8]
            \coordinate (a) at (0.6, 0);
            \coordinate (b) at (-0.6, 0);
            \node (o) [empty] at (0, 0) {};
            \draw [double line] (b) -- (a);
            \draw (b) + (120:1) -- (a) -- +(-60:1);
            \draw (a) + (60:1) -- (b) -- +(-120:1);
        \end{tikzpicture}
        \qquad + \qquad
        \begin{tikzpicture}[baseline = (o.south), scale = 0.8]
            \node (o) [empty] at (0, 0) {};
            \draw (-1, 1) -- (1, -1);
            \draw (-1, -1) -- (1, 1);
        \end{tikzpicture}
    \end{equation*}
    Show that, at threshold ($\vec{p}_i = 0$), these diagrams sum to \textit{zero}. (Hint: It may be easier to first consider
    the specific process $\pi^1\pi^1 \to \pi^2\pi^2$, for which only the first and fourth diagrams are nonzero, before tackling 
    the general case.) Show that, in the special case $N = 2$ (1 species of pion), the term of $\bigo(p^2)$ also cancels.

    \item Add to $V$ a symmetriy-breaking term,
    \begin{equation*}
        \Delta V = -a\Phi^N,
    \end{equation*}
    where $a$ is a (small) constant. (In QCD, a term of this form is produces if the $u$ and $d$ quarks have the same nonvanishing
    mass.) Find the new value of $v$ that minimizes $V$, and work out the content of the theory about that point. Show that the
    pion acquires a mass such that $m_\pi^2 \sim a$, and show that the pion scattering amplitude at threshold is now nonvanishing
    and also proportional to $a$.
\end{problembody}

\solution
\begin{problembody}
    \item The propagator for this Hamitonian is obvious. Vertex rule can be obtained by considering
    \begin{equation}\label{equ:cp3:vertex_contract}
        \bra{0}T \left[ \Phi^k(x_k)\Phi^l(x_l)\Phi^i(x_i)\Phi^j(x_j) 
            \int \dd^4x \, \frac{-i\lambda}{4} 
            \sum_{m = 1}^N \sum_{n = 1}^N (\Phi^m)^2(\Phi^n)^2
        \right]\ket{0},
    \end{equation}
    and only consider connected part, i.e. the four $\Phi^i$'s outside the integration shouldn't contract with each other. The interchange
    of the $\Phi^m$ and $\Phi^n$ gives a factor of two, while the contraction of the four $\Phi^i$'s with $\Phi^m$ and $\Phi^n$ gives 2, 
    respectively. Thus \eqref{equ:cp3:vertex_contract} becomes
    \begin{equation*}
        -2i\lambda \left(
            \delta^{ij}\delta^{kl} 
            + \delta^{ik}\delta^{jl}
            + \delta^{il}\delta^{jk}    
        \right)
        \int \dd^4x \, D_F(x_k - x)D_F(x_l - x)D_F(x_i - x)D_F(x_j - x).
    \end{equation*}
    So that the vertex rule is given by
    \begin{equation*}
        \begin{tikzpicture}[baseline = (o.south)]
            \node (k) at (-1,  1) {$k$};
            \node (l) at ( 1,  1) {$l$};
            \node (i) at (-1, -1) {$i$};
            \node (j) at ( 1, -1) {$j$};
            \draw (k) -- (0, 0) -- (l);
            \draw (i) -- (0, 0) -- (j);
            \node (o) [vertex] at (0, 0) {};
        \end{tikzpicture}
        = -2i\lambda(
            \delta^{ij}\delta^{kl}
            + \delta^{il}\delta^{jk}
            + \delta^{ik}\delta^{jl}
        ).
    \end{equation*}
    Thus
    \begin{equation*}
        i\ims(\Phi^i\Phi^j \to \Phi^k\Phi^l) = -2i\lambda \left(
            \delta^{ij}\delta^{kl}
            + \delta^{il}\delta^{jk}
            + \delta^{ik}\delta^{jl}
        \right) \equiv -2i\lambda A,
    \end{equation*}
    and the cross section
    \begin{equation*}
        \df{\sigma}{\Omega}(\Phi^i\Phi^j \to \Phi^k\Phi^l)
        = \frac{\abs{\ims}^2}{64\pi^2 E_{CM}^2}
        = \frac{\lambda^2 A^2}{16 \pi^2 E_{CM}^2}.
    \end{equation*}
    For the three processes in concern, we have $A = 1, 1, 3$ respectively. As we can see, all of them are isotropic.

    \item In this case, $V$ is given by
    \begin{equation*}
        V = -\frac{1}{2}\mu^2 (\Phi^i)^2 + \frac{\lambda}{4}\left((\Phi^i)^2\right)^2,
    \end{equation*}
    which is quadratic in $(\Phi^i)^2$, and has minimum at $(\Phi^i)^2 = \mu^2 / \lambda$. Thus $v = \mu / \sqrt{\lambda}$. Then
    $V$ becomes
    \begin{align*}
        V & = \frac{\lambda}{4} \left[ (\pi^i)^2 + (v + \sigma)^2\right]^2
        - \frac{1}{2}\mu^2 \left[ (\pi^i)^2 + (v + \sigma)^2 \right]\\
        %
        & = \frac{\lambda}{4}\left((\pi^i)^2\right)^2 + \frac{\lambda}{4}\sigma^4 
        + \sqrt{\lambda} \mu (\pi^i)^2\sigma 
        + \frac{\lambda}{2}(\pi^i)^2\sigma^2
        + \sqrt{\lambda}\mu \sigma^3
        + \mu^2\sigma^2
        - \frac{\mu^4}{4\lambda}.
    \end{align*}
    There is no $(\pi^i)^2$ term, so the $\pi^i$'s are massless, while the term $\mu^2\sigma^2$ indicates $\sigma$ field has mass $\sqrt{2}\mu$, thus the 
    propagators are
    \begin{equation*}
        \wick{\c \pi^i \c \pi^j} = \frac{i\delta^{ij}}{k^2 + i\epsilon}, \qquad
        \wick{\c \sigma \c \sigma} = \frac{i}{k^2 - 2\mu^2 + i\epsilon}.
    \end{equation*}
    Feynman rule can be read directly from $V$:
    \begin{equation*}
        \begin{tikzpicture}[baseline = (o.south), scale = 0.6]
            \draw (135:1) -- (0, 0) -- (45:1);
            \draw (-135:1) -- (0, 0) -- (-45:1);
            \node at (135:1.3) {$k$};
            \node at (45:1.3) {$l$};
            \node at (-135:1.3) {$i$};
            \node at (-45:1.3) {$j$};
            \node (o) [vertex] at (0, 0) {};
        \end{tikzpicture}
        = -2i\lambda\left(
            \delta^{kl}\delta^{ij}
            + \delta^{ki}\delta^{lj}
            + \delta^{kj}\delta^{li}    
        \right),
        \qquad
        \begin{tikzpicture}[baseline = (o.south), scale = 0.6]
            \draw [double line] (135:1) -- (0, 0) -- (45:1);
            \draw [double line] (-135:1) -- (0, 0) -- (-45:1);
            \node (o) [vertex] at (0, 0) {};
        \end{tikzpicture} = -6i\lambda,
    \end{equation*}
    \begin{equation*}
        \begin{tikzpicture}[baseline = (o.south), scale = 0.6]
            \draw [double line] (0, 0) -- (0, 1);
            \draw (0, 0) -- (-150:1);
            \draw (0, 0) -- (-30:1);
            \node at (-150:1.3) {$i$};
            \node at (-30:1.3) {$j$};
            \node (o) [vertex] at (0, 0){};
        \end{tikzpicture} = -2i\sqrt{\lambda}\mu \delta^{ij},
        \qquad
        \begin{tikzpicture}[baseline = (o.south), scale = 0.6]
            \draw [double line] (0, 0) -- (0, 1);
            \draw [double line] (-30:1) -- (0, 0) -- (-150:1);
            \node (o) [vertex] at (0, 0){};
        \end{tikzpicture} = -6i\sqrt{\lambda}\mu,
        \qquad
        \begin{tikzpicture}[baseline = (o.south), scale = 0.6]
            \draw [double line] (135:1) -- (0, 0) -- (45:1);
            \draw (-135:1) -- (0, 0) -- (-45:1);
            \node at (-135:1.3) {$i$};
            \node at (-45:1.3) {$j$};
            \node (o) [vertex] at (0, 0) {};
        \end{tikzpicture} = -2i\lambda\delta^{ij}.
    \end{equation*}

    \item We evaluate each diagram:
    \begin{align*}
        \begin{tikzpicture}[baseline = (o.south)]
            \coordinate (a) at (0, 0.6);
            \coordinate (b) at (0, -0.6);
            \node (o) [empty] at (0, 0) {};
            \draw [double line] (b) -- (a);
            \draw (a) -- +(150:1) [arrowed, text = $p_3$];
            \draw (a) -- +(30:1) [arrowed, texti = $p_4$];
            \node at ($ (a) + (150:1.3) $) {$k$};
            \node at ($ (a) + (30:1.3) $) {$l$};
            \draw ($ (b) + (-150:1) $) -- (b) [arrowed, text = $p_1$];
            \draw ($ (b) + (-30:1) $) -- (b) [arrowed, texti = $p_2$];
            \node at ($ (b) + (-150:1.3) $) {$i$};
            \node at ($ (b) + (-30:1.3) $) {$j$};
        \end{tikzpicture}
        & = -4\lambda\mu^2\delta^{ij}\delta^{kl}\frac{i}{(p_1 + p_2)^2 - 2\mu^2},\\
        %
        \begin{tikzpicture}[baseline = (o.south)]
            \coordinate (a) at (-0.6, 0);
            \coordinate (b) at (0.6, 0);
            \node (o) [empty] at (0, 0) {};
            \draw [double line] (b) -- (a);
            \draw (a) -- +(120:1) [arrowed, texti = $p_3$];
            \draw (b) -- +(60:1) [arrowed, text = $p_4$];
            \node at ($ (a) + (120:1.3) $) {$k$};
            \node at ($ (b) + (60:1.3) $) {$l$};
            \draw ($ (a) + (-120:1) $) -- (a) [arrowed, texti = $p_1$];
            \draw ($ (b) + (-60:1) $) -- (b) [arrowed, text = $p_2$];
            \node at ($ (a) + (-120:1.3) $) {$i$};
            \node at ($ (b) + (-60:1.3) $) {$j$};
        \end{tikzpicture}
        & = -4\lambda\mu^2\delta^{ki}\delta^{lj}\frac{i}{(p_3 - p_1)^2 - 2\mu^2},\\
        \begin{tikzpicture}[baseline = (o.south)]
            \coordinate (a) at (-0.6, 0);
            \coordinate (b) at (0.6, 0);
            \node (o) [empty] at (0, 0) {};
            \draw [double line] (a) -- (b);
            \draw (a) -- node[midway, label=above:$p_4$] {} ($ (b) + (60:1) $) [arrowed];
            \draw (b) -- node[midway, label=above:$p_3$] {} ($ (a) + (120:1) $) [arrowed];
            \node at ($ (a) + (120:1.3) $) {$k$};
            \node at ($ (b) + (60:1.3) $) {$l$};
            \draw ($ (a) + (-120:1) $) -- (a) [arrowed, texti = $p_1$];
            \draw ($ (b) + (-60:1) $) -- (b) [arrowed, text = $p_2$];
            \node at ($ (a) + (-120:1.3) $) {$i$};
            \node at ($ (b) + (-60:1.3) $) {$j$}; 
        \end{tikzpicture}
        & = -4\lambda\mu^2\delta^{li}\delta^{kj}\frac{i}{(p_4 - p_1)^2 - 2\mu^2},\\
        \begin{tikzpicture}[baseline = (o.south)]
            \draw (135:1) -- (0, 0) -- (-45:1);
            \draw (45:1) -- (0, 0) -- (-135:1);
            \node (o) [empty] at (0, 0) {};
            \node at (135:1.3) {$k$};
            \node at (45:1.3) {$l$};
            \node at (-45:1.3) {$j$};
            \node at (-135:1.3) {$i$};
        \end{tikzpicture}
        & = -2i\lambda\left(
            \delta^{kl}\delta^{ij}
            + \delta^{ki}\delta^{lj}
            + \delta^{kj}\delta^{li}    
        \right).
    \end{align*}
    The amplitude is the sum of the above diagrams:
    \begin{equation*}
        i\ims = -2i\lambda\delta^{ij}\delta^{kl}\frac{s}{s - 2\mu^2}
        - 2i\lambda\delta^{ik}\delta^{lj}\frac{t}{t - 2\mu^2}
        - 2i\lambda\delta^{il}\delta^{kj}\frac{u}{u - 2\mu^2},
    \end{equation*}
    where 
    \begin{equation*}
        s = (p_1 + p_2)^2, \qquad t = (p_3 - p_1)^2, \qquad u = (p_4 - p_1)^2.
    \end{equation*}
    At threshold where $\vec{p}_i = 0$, we have $s = t = u = 0$, so $\ims$ vanishes. In the special case $N = 2$, all
    the $\delta$'s in $\ims$ become factors of one:
    \begin{equation*}
        \ims = -2\lambda\left(
            \frac{s}{s - 2\mu^2}
            + \frac{t}{t - 2\mu^2}
            + \frac{u}{u - 2\mu^2}    
        \right).
    \end{equation*}

    \item With the presence of $\Delta V$, $V$ is now
    \begin{equation*}
        V = \frac{\lambda}{4} \left[
            (\pi^i)^2 + (\Phi^N)^2
        \right]^2
        - \frac{1}{2}\mu^2\left[
            (\pi^i)^2 + (\Phi^N)^2    
        \right]
        - a \Phi^N.
    \end{equation*}
    Let $v$ be the value of $\Phi^N$ that minimizes $V$ when $\pi^i = 0$. The condition is
    \begin{equation*}
        \pd{V}{v} = \lambda v^3 - \mu^2 v - a = 0.
    \end{equation*}
    Since $a$ is small, $v$ can be solved by perturbation. Let $v = \mu / \sqrt{\lambda} + xa + o(a)$, with $x$ be a constant to be
    determined. Substitude into the equation above, we get $x = 1 / 2\mu^2$. The new potential is then
    \begin{align*}
        V & = \frac{\lambda}{4}\left[
            (\pi^i)^2 + (v + \sigma)^2    
        \right] - \frac{1}{2}\mu^2 \left[
            (\pi^i)^2 + (v + \sigma)^2
        \right] - a (v + \sigma)\\
        %
        & = \frac{a\sqrt{\lambda}}{2\mu} (\pi^i)^2 
        + \left(\mu^2 + \frac{3a\sqrt{\lambda}}{2\mu}\right)\sigma^2
        + \left(\frac{a\lambda}{2\mu^2} + \sqrt{\lambda}\mu\right)(\pi^i)^2\sigma
        + \left(\frac{a\lambda}{2\mu^2} + \sqrt{\lambda}\mu\right) \sigma^3\\
        & \qquad + \frac{\lambda}{4}\left((\pi^i)^2\right)^2
        + \frac{\lambda}{2}(\pi^i)^2\sigma^2 + \frac{\lambda}{4}\sigma^4
        -\frac{\mu^4}{4\lambda} - \frac{a\mu}{\sqrt{\lambda}} - o(a).
    \end{align*}
    The $\Delta V$ modifies the particle masses, the $\pi^2\sigma$ and $\sigma^3$ vertices. The pion acquires mass $m_\pi^2 = a\sqrt{\lambda} / \mu \propto a$.
    Since the $\pi^4$ vertex is unaffected, while the coeffecient of $\pi^2\sigma$ vertex is added by $a\lambda / 2\mu^2$, the pion scattering amplitude at threshold 
    now no longer vanish:
    \begin{align*}
        i\ims & = -4i\lambda\left(\mu^2 + \frac{\sqrt{\lambda}}{\mu}a\right)\left(
            \frac{\delta^{ij}\delta^{kl}}{s - m_\sigma^2} 
            + \frac{\delta^{ik}\delta^{lj}}{t - m_\sigma^2} 
            + \frac{\delta^{il}\delta^{kj}}{u - m_\sigma^2}
        \right)\\
        & \qquad -2i\lambda\left(
            \delta^{kl}\delta^{ij}
            + \delta^{ki}\delta^{lj}
            + \delta^{kj}\delta^{li}    
        \right)\\
        %
        & = i\ims_0 + 4i\frac{\lambda\sqrt{\lambda}}{\mu}a \left(
            \frac{\delta^{ij}\delta^{kl}}{s - m_\sigma^2} 
            + \frac{\delta^{ik}\delta^{lj}}{t - m_\sigma^2} 
            + \frac{\delta^{il}\delta^{kj}}{u - m_\sigma^2}
        \right),
    \end{align*}
    where $m_\sigma$ is the mass of ``sigmaon'' with the presence of $\Delta V$, and $\ims$ is the original scattering amplitude with the sigmaon mass in the propagator 
    replaced by $m_\sigma$. Thus $\ims$ vanishes at threshold and the remaining terms are proportional to $a$.
\end{problembody}

\problem \textbf{Rutherford scattering.} The cross section for scattering of an electron by the Coulomb field of a nucleus can be computed, to lowest 
order, without quantizing the electromagnetic field. Insteadm treat the field as a given , classical potential $A_\mu(x)$. The interaction Hamitonian
is 
\begin{equation*}
    H_I = \int \dd^3 x \, e\bar{\psi}\gamma^\mu\psi A_\mu,
\end{equation*}
where $\psi(x)$ is the usual quantized Dirac field.
\begin{problembody}
    \item Show that the \textit{T}-matrix element for electron scattering of a localized classical potential is, to lowest order,
    \begin{equation*}
        \bra{p'}iT\ket{p} = -ie\bar{u}(p')\gamma^\mu u(p) \tilde{A}_\mu(p' - p),
    \end{equation*}
    where $\widetilde{A}_\mu(q)$ is the four dimensional Fourier transform of $A_\mu(x)$.

    \item If $A_\mu(x)$ is time independent, its Fourier transform contains a delta function of energy. It is then natural to define
    \begin{equation*}
        \bra{p'}iT\ket{p} \equiv i\ims (2\pi) \delta(E_f - E_i),
    \end{equation*}
    where $E_i$ and $E_f$ are the initial and final energies of the particle, and to adopt a new Feynman rule for computing $\ims$:
    \begin{equation*}
        \feynmandiagram[horizontal = a to b, baseline = (a.base)]{
            i1 -- [fermion] a [dot] -- [fermion] i2,
            b [crossed dot] -- [photon] a
        };
        = -ie\gamma^\mu\widetilde{A}_\mu(\vec{q}),
    \end{equation*}
    where $\widetilde{A}_\mu(\vec{q})$ is the three dimensional Fourier transform of $A_\mu(x)$. Given this definition of $\ims$, show that the cross section 
    for scattering off a time-independent, localized potential is
    \begin{equation*}
        \dd\sigma = \frac{1}{v_i}\frac{1}{2E_i}\frac{\dd^3p_f}{(2\pi)^3}
        \frac{1}{2E_f}\abs{\ims(p_i \to p_f)}^2 (2\pi)\delta(E_f - E_i),
    \end{equation*}
    where $v_i$ is the particle's initial velocity. This formula is a natural modification of (4.79). Integrate over $\abs{p_f}$ to find a simple expression for $\dd\sigma / \dd \Omega$.

    \item Specialize to the case of electron scattering from a Coulomb potential ($A^0 = Ze / 4\pi r$). Working in the nonrelativistic limit, derive the Rutherford formula,
    \begin{equation*}
        \df{\sigma}{\Omega} = \frac{\alpha^2 Z^2}{4m^2v^4\sin^4(\theta / 2)}.
    \end{equation*}
    (With a few calculational tricks from Section 5.1, you will have no difficulty evaluating the general cross section in the relativistic case; see Problem 5.1.)
\end{problembody}

\solution
\begin{problembody}
    \item \begin{align*}
        \bra{p'}iT\ket{p} & = -ie \int \dd^4x \, 
        \wick{\langle\c p'| \c{\bar\psi}\gamma^\mu\c\psi A_\mu(x) |\c p\rangle}\\
        %
        & = -ie \int \dd^4x \, \bar{u}(p')\slashed{A}(x)u(p) e^{ix\cdot(p' - p)}\\
        & = -ie \bar u(p') \gamma^\mu u(p) \widetilde{A}_\mu(p' - p).
    \end{align*}

    \item The cross section can be derived the same way as (4.79), with
    \begin{equation*}
        \ket{\text{in}} = \int \frac{\dd^3p}{(2\pi)^3} \frac{\phi_i(\vec p)e^{-i\vec b \cdot \vec p}}{\sqrt{2E_p}} \ket{p}_{\text{in}}, \qquad \text{and} \; 
        \ket{\text{out}} = \frac{1}{\sqrt{(2\pi)^32E_f}}\ket{p_f}_{\text{out}}
    \end{equation*}
    being the inital and final states instead, where $\phi_i$ have narrow peak at $p_i$. The cross section is the transition probability integrate over
    the `impact parameter' $\vec b$:
    \begin{align*}
        \frac{\dd\sigma}{\dd^3p_f} & = \int \dd^2b \, \abs{\inner{\text{in}}{\text{out}}}^2\\
        & = \int \dd^2 b \int \frac{\dd^3p_1}{(2\pi)^3} 
        \int \frac{\dd^3p_2}{(2\pi)^3}
        \frac{
            \phi_i^\ast(\vec p_1)
            \phi_i(\vec p_2)
            e^{-i\vec b \cdot (\vec p_2 - \vec p_1)}
        }{
            \sqrt{2E_1}\sqrt{2E_2}(2\pi)^3 2E_f
        }(2\pi)^2\abs \ims^2 \delta(E_1 - E_f) \delta(E_2 - E_f)\\
        & = \frac{1}{(2\pi)^5 2E_f}\int \dd^3p_1 \int \dd^3p_2
        \frac{\phi_i^\ast(\vec p_1)\phi_i(\vec p_2)}{2E_1} \abs \ims^2
        \delta(E_1 - E_2)\delta^2(\vec p_{1\bot} - \vec p_{2\bot})\delta(E_1 - E_f)\\
        & = \frac{1}{(2\pi)^5 2E_f}
        \int \dd^3p_1 \int \dd^3p_2
        \frac{\phi_i^\ast(\vec p_1)\phi_i(\vec p_2)}{2E_1 v_1} \abs \ims^2
        \delta^3(\vec p_1 - \vec p_2)\delta(E_1 - E_f)\\
        & = \frac{1}{(2\pi)^5 2E_f}
        \int \dd^3p_1
        \frac{\abs{\phi_i(\vec p_1)}^2}{2E_1 v_1} \abs \ims^2 \delta(E_1 - E_f)\\
        & = \frac{1}{(2\pi)^2 2E_f 2E_i v_i}\abs\ims^2 \delta(E_i - E_f).
    \end{align*}
    Integrate over modulu,
    \begin{align*}
        \df{\sigma}{\Omega} & = \frac{\abs\ims^2}{16\pi^2 \abs{p_i}} 
        \int_0^\infty \frac{p^2}{2E_f}\delta(E_i - E_f) \dd p\\
        & = \frac{\abs\ims^2}{32\pi^2}.
    \end{align*}

    \item The amplitude in the lowest order is
    \[
        i\ims = -ie\bar u(p_f)\gamma^\mu u(p_i)\widetilde A_\mu(\vec p_f - \vec p_i),
    \]
    where $\widetilde{A}_\mu(\vec q)$ can be calculated below
    \begin{align*}
        \widetilde{A}^0(\vec q) & = \int \dd^3x \, \frac{Ze}{4\pi r} e^{-i\vec q\cdot\vec x}\\
        & = \frac{Ze}{2}\int_0^\pi \dd(-\cos\theta)\int_0^\infty r e^{-iqr\cos\theta} \dd r\\
        & = \frac{Ze}{-2iq}\int_0^\infty e^{-\epsilon r}\left(e^{iqr} - e^{-iqr}\right) \dd r\\
        & = \frac{Ze}{q^2 + \epsilon^2},
    \end{align*}
    and the spacial components are zero. The amplitude then reads
    \begin{align*}
        i\ims & = -ie \bar u(p_f)\gamma^0u(p_i) \frac{Ze}{\abs{\vec p_f - \vec p_i}^2}\\
        & = -2im \delta^{ss'} \frac{Ze^2}{\abs{\vec p_f - \vec p_i}^2}\\
        & = \frac{-i\delta^{ss'}Ze^2}{2mv^2\sin^2(\theta / 2)}.
    \end{align*}
    Thus
    \[
        \df{\sigma}{\Omega} = \frac{e^4Z^2}{64\pi^2 m^2 v^4 \sin^4(\theta / 2)}
        = \frac{\alpha^2Z^2}{4m^2 v^4 \sin^4(\theta / 2)}.    
    \]
\end{problembody}