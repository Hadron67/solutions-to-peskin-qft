\section{Chapter 3}

\setcounter{equation}{150}
\problem \textbf{Lorentz group.} Recall from Eq. (3.17) the Lorentz commutation relations,
\begin{equation*}
    [J^{\mu\nu}, J^{\rho\sigma}] = i(g^{\nu\rho}J^{\mu\sigma} - g^{\mu\rho}J^{\nu\sigma} - g^{\nu\sigma}J^{\mu\rho} + g^{\mu\sigma}J^{\nu\rho}).
\end{equation*}
\begin{problembody}
    \item \label{ex:cp3:lor_gen_com} 
    Define the generators of rotations and boosts as
    \begin{equation*}
        L^i = \frac{1}{2}\epsilon^{ijk} J^{jk}, \quad K^i = J^{0i},
    \end{equation*}
    where $i, j, k = 1, 2, 3$. An infinitesimal Lorentz transformation can then be written
    \begin{equation}\label{equ:cp2:infsm_lor_trans}
        \Phi \to \left(1 - i\vec{\theta}\cdot\vec{L} - i\vec{\beta}\cdot\vec{K}\right)\Phi.
    \end{equation}
    Write the commutation relations of these operators explicitly. (For example, $[L^i, L^j] = i\epsilon^{ijk}L^k$)
    Show that the combinations 
    \begin{equation*}
        \vec{J}_{+} = \frac{1}{2} \left(\vec{L} + i\vec{K}\right) \quad \text{and} 
        \quad \vec{J}_{-} = \frac{1}{2} \left(\vec{L} - i\vec{K}\right)
    \end{equation*} 
    commute with one another and separately satisfy the commutation relations of angular momentum.

    \item The finite-dimensional representations of the rotation group correspond precisely to the allowed values for 
    angular momentum: integers or half-integers. The result of part \ref{ex:cp3:lor_gen_com} implies that all finite-dimensional
    representations of the Lorentz group correspond to pairs of integers or half integers, $(j_+, j_-)$, corresponding to pairs 
    of representations of the rotation group. Using the fact that $\vec{J} = \vec{\sigma} / 2$ in the spin-laws of the 2-component
    objects transforming according to the $(\frac{1}{2}, 0)$ and $(0, \frac{1}{2})$ representations of the Lorentz group. Show 
    that these correspond precisely to the transformations of $\psi_L$ and $\psi_R$ given in (3.27).

    \item The identity $\vec{\sigma}^T = -\sigma^2 \vec{\sigma}\sigma^2$ allows us to rewrite the $\psi_L$ transformation in the 
    unitarily equivalent form
    \begin{equation*}
        \psi' \to \psi' \left(1 + i\vec{\theta}\cdot\frac{\vec{\sigma}}{2} + \vec{\beta}\cdot\frac{\vec{\sigma}}{2}\right),
    \end{equation*}
    where $\psi' = \psi_L^T \sigma^2$. Using this law, we can represent the object that transforms as $(\frac{1}{2}, \frac{1}{2})$
    as a $2 \times 2$ matrix that has the $\psi_R$ transformation law on the left and, simutaneously, the transposed $\psi_L$ transformation
    on the right. Parametrize this matrix as
    \begin{equation*}
        \begin{pmatrix}
            V^0 + V^3  & V^1 - iV^2 \\
            V^1 + iV^2 & V^0 - V^3
        \end{pmatrix}.
    \end{equation*}
    Show that the object $V^\mu$ transformations as a 4-vector.
\end{problembody}

\solution
\begin{problembody}
    \item There are 3 sets of commutators. Commutators of the rotation generators are
    \begin{align*}
        [L^i, L^j] & = \frac{1}{4} \epsilon^{ikl} \epsilon^{jmn} [J^{kl}, J^{mn}]\\
        & = \frac{i}{4} \epsilon^{ikl} \epsilon^{jmn} \left(g^{lm}J^{kn} + g^{kn}J^{lm} - g^{km}J^{ln} - g^{ln}J^{km}\right)\\
        & = \frac{i}{4} \left(-\epsilon^{ikm}\epsilon^{jmn}J^{kn} - \epsilon^{ikl}\epsilon^{jmk}J^{lm} 
        + \epsilon^{ikl}\epsilon^{jkn}J^{ln} + \epsilon^{ikl}\epsilon^{jml}J^{km}\right)\\
        & = i J^{ij}\\
        & = i \epsilon^{ijk} L^k.
    \end{align*}
    Commutators of the boost generators are
    \begin{align*}
        [K^i, K^j] & = [J^{0i}, J^{0j}]\\
        & = i\left(g^{i0}J^{0j} + g^{0j}J^{i0} - g^{00}J^{ij} - g^{ij}J^{00}\right)\\
        & = -iJ^{ij}\\
        & = -i \epsilon^{ijk} L^k.
    \end{align*}
    Commutators between rotation generators and boost generators are
    \begin{align*}
        [L^i, K^j] & = \frac{1}{2} \epsilon^{ikl} [J^{kl}, J^{0j}]\\
        & = \frac{i}{2} \epsilon^{ikl} \left(g^{l0}J^{kj} + g^{kj}J^{l0} - g^{k0}J^{lj} - g^{lj}J^{k0}\right)\\
        & = i\epsilon^{ijk} J^{0k}\\
        & = i\epsilon^{ijk} K^j.
    \end{align*}
    And the commutation relations of $\vec{J}_\pm$ are
    \begin{align*}
        [J^i_\pm, J^j_\pm] & = \frac{1}{4}[L^i \pm iK^i, L^j \pm iK^j]\\
        & = \frac{1}{4} \left(i\epsilon^{ijk}L^k \mp 2i\epsilon^{ijk}K^k + i\epsilon^{ijk}L^k\right)\\
        & = \frac{i}{2} \epsilon^{ijk} \left(L^k \pm iK^k\right)\\
        & = i\epsilon^{ijk} J^k_\pm.
    \end{align*}
    That is, $\vec{J}_\pm$ satisfy the commutation relations of angular momentum. Further more, the commutators between 
    $\vec{J}_+$ and $\vec{J}_-$ are
    \begin{align*}
        [J^i_+, J^j_-] & = \frac{1}{4} [L^i + iK^i, L^j - iK^j]\\
        & = \frac{1}{4} \left(i\epsilon^{ijk}L^k + \epsilon^{ijk}K^k - \epsilon^{ijk}K^k -i\epsilon^{ijk}L^k\right)\\
        & = 0.
    \end{align*}
    So $\vec{J}_+$ and $\vec{J}_-$ commutes with each other.

    \item The infinitesimal transformation \eqref{equ:cp2:infsm_lor_trans} can now be written, in terms of $\vec{J}_\pm$, as
    \begin{equation}\label{equ:cp3:lor_uni_trans}
        \Phi \to \left[1 - i\left(\vec{\theta} - i\vec{\beta}\right)\cdot\vec{J}_+ - i\left(\vec{\theta} + i\vec{\beta}\right)\cdot\vec{J}_-\right] \Phi.
    \end{equation}
    For $(\frac{1}{2}, 0)$, make substitution
    \begin{equation*}
        \vec{J}_+ \to \frac{\vec{\sigma}}{2}, \quad \vec{J}_- \to \vec{0},
    \end{equation*}
    so that the transformation is
    \begin{equation}\label{equ:cp3:spinor_trans_L}
        \psi \to \left(1 - i\vec{\theta}\cdot\frac{\vec{\sigma}}{2} - \vec{\beta}\cdot\frac{\vec{\sigma}}{2}\right)\psi,
    \end{equation}
    which is the same as the transformation of $\psi_L$ in (3.37). Similarly, for $(0, \frac{1}{2})$, the transformation is
    \begin{equation*}
        \psi \to \left(1 - i\vec{\theta}\cdot\frac{\vec{\sigma}}{2} + \vec{\beta}\cdot\frac{\vec{\sigma}}{2}\right) \psi,
    \end{equation*}
    which is the transformation of $\psi_R$.

    \item The matrix can be written as $V^0 + \vec{\sigma}\cdot\vec{V}$. Let $\vec{\alpha} = \vec{\theta} - i\vec{\beta}$,
    Then the (infinitesimal) transformation for the matrix is
    \begin{align*}
        V^0 + \vec{\sigma}\cdot\vec{V} & \to 
        \left(1 - i\vec{\alpha}^\ast\cdot\frac{\vec{\sigma}}{2}\right)
        \left(V^0 + \vec{\sigma}\cdot\vec{V}\right)
        \left(1 + i\vec{\alpha}\cdot\frac{\vec{\sigma}}{2}\right)\\
        & = \left(1 + \vec{\beta}\cdot\vec{\sigma}\right) V^0 + \vec{\sigma}\cdot\vec{V} + \vec{\beta}\cdot\vec{V} + \left(\vec{\theta}\times\vec{V}\right)\cdot\vec{\sigma}\\
        & = V^0 + \vec{\beta}\cdot\vec{V} + \left(\vec{V} + \vec{\beta}V^0 + \vec{\theta}\times\vec{V}\right)\cdot\vec{\sigma}.
    \end{align*}
    Thus, the infinitesimal transformation of $V^\mu$ is
    \begin{subequations}\label{equ:cp3:v_trans}
        \begin{align}
            V^0 & \to V^0 + \vec{\beta}\cdot\vec{V};\\
            \vec{V} & \to \vec{V} + \vec{\beta}V^0 + \vec{\theta}\times\vec{V}.
        \end{align}
    \end{subequations}
    This is exactly the transformation law for 4-vectors. To make it more obvious, let
    \begin{subequations}\label{equ:cp3:omega_def}
        \begin{align}
            \omega_{0i} & = \beta_i, \\
            \omega_{ij} & = \epsilon_{ijk} \theta^k.
        \end{align}
    \end{subequations}
    So that \eqref{equ:cp3:v_trans} can be written in 4-vector form as
    \begin{equation*}
        V^\mu \to V^\mu + \omega^\mu{}_\nu V^\nu.
    \end{equation*}
    Which is the same as (3.19) by setting $(\mathcal{J}^{\mu\nu})_{\alpha\beta} = 2i \delta^{[\mu}_\alpha \delta^{\nu]}_\beta$.
\end{problembody}

\problem Derive the \textit{Gordon identity},
\begin{equation}\label{equ:cp3:gordon_identity}
    \bar{u}(p')\gamma^\mu u(p) = \bar{u}(p') \left[\frac{p'^{\mu} + p^\mu}{2m} + \frac{i\sigma^{\mu\nu}q_\nu}{2m}\right] u(p)
\end{equation}
where $q = (p' - p)$. We will put this formula to use in Chapter 6.

\solution
\begin{align*}
    \bar{u}(p')\frac{i\sigma^{\mu\nu}q_\nu}{2m} u(p) 
    & = -\frac{1}{4m} \bar{u}(p') \left[\gamma^\mu(\slashed{p}' - \slashed{p}) - (\slashed{p}' - \slashed p) \gamma^\mu \right] u(p)\\
    & = -\frac{1}{4m} \bar{u}(p') \left[-2\slashed{p}'\gamma^\mu - 2\gamma^\mu\slashed{p} + 2(p'^\mu + p^\mu) \right] u(p)\\
    & = \bar{u}(p')\gamma^\mu u(p) - \bar{u}(p')\frac{p'^\mu + p^\mu}{2m}u(p).
\end{align*}
Moving the second term of RHS to LHS, we obtain \eqref{equ:cp3:gordon_identity}.

\problem \textbf{Spinor products.} (This problem, together with Problems 5.3 and 5.6, introduces an efficient computational method for
processes involving massless particles.) Let $k_0^\mu$, $k_1^\mu$ be fixed 4-vectors satisfying $k_0^2 = 0, k_1^2 = -1, k_0\cdot k_1 = 0$.
Define basic spinors in the following way: Let $u_{L0}$ be the left-handed spinor for a fermion with momentum $k_0$. Let $u_{R0} = \slashed{k}_1 u_{L0}$.
Then, for any $p$ such that $p$ is lightlike ($p^2 = 0$), define
\begin{equation*}
    u_L(p) = \frac{1}{\sqrt{2p\cdot k_0}} \slashed{p} u_{R0}, \qquad u_R(p) = \frac{1}{\sqrt{2p\cdot k_0}}  \slashed{p} u_{L0}.
\end{equation*}
This set of convensions define defines the phase of spinors unambiguously (except when $p$ is paralle to $k_0$).
\begin{problembody}
    \item Show that $\slashed{k}_0 u_{R0} = 0$. Show that, for any lightlike $p$, $\slashed{p}u_L(p) = \slashed{p}u_R(p) = 0$.
    \item \label{ex:cp3:sp_choices} For choices $k_0 = (E, 0, 0, -E)$, $k_1 = (0, 1, 0, 0)$, construct $u_{L0}$, $u_{R0}$, $u_L(p)$, and $u_R(p)$ explicitly.
    \item Define the \textit{spinor products} $s(p_1, p_2)$ and $t(p_1, p_2)$, for $p_1$, $p_2$ lightlike, by
    \begin{equation*}
        s(p_1, p_2) = \bar{u}_R(p_1) u_L(p_2), \qquad t(p_1, p_2) = \bar{u}_L(p_1) u_R(p_2).
    \end{equation*}
    Using the explicit forms for the $u_\lambda$ given in part \ref{ex:cp3:sp_choices}, compute the spinor products explicitly and show that
    $t(p_1, p_2) = (s(p_2, p_1))^\ast$ and $s(p_1, p_2) = -s(p_2, p_1)$. In addition, show that
    \begin{equation*}
        \left| s(p_1, p_2) \right|^2 = 2p_1 \cdot p_2.
    \end{equation*}
    Thus the spinor products are the square roots of 4-vector dot products.
\end{problembody}

\solution
\begin{problembody}
    \item Since $\slashed{k}_0 u_{L0} = 0$, so that
    \begin{equation*}
        \slashed{k}_0 u_{R0} = \slashed{k}_0 \slashed{k}_1 u_{L0} = \left(2k_0 \cdot k_1 - \slashed{k}_1 \slashed{k}_0\right) u_{L0} = 0.
    \end{equation*}
    The next two equations are obvious, since $\slashed{p}\slashed{p} = p^2 = 0$.

    \item A left-handed massless particle with the given momentum $k_0$ has $\xi = (1 \, 0)^T$. Using (3.50) we get
    \begin{equation*}
        u_{L0}(k_0) = \sqrt{2E} \begin{pmatrix}
            1\\ 0\\ 0\\ 0
        \end{pmatrix}
    \end{equation*}
    and
    \begin{equation*}
        u_{R0}(k_0) = \slashed{k}_1 u_{L0}(k_0) = \begin{pmatrix}
            0 & -\sigma_x \\
            \sigma_x & 0
        \end{pmatrix} u_{L0}(k_0) = \sqrt{2E}\begin{pmatrix}
            0 \\ 0 \\ 0 \\ 1
        \end{pmatrix},
    \end{equation*}
    \begin{equation*}
        u_L(p) = \frac{1}{\sqrt{E_p + p_z}} \begin{pmatrix}
            0 & E_p - \vec{p}\cdot\vec{\sigma} \\
            E_p + \vec{p}\cdot\vec{\sigma} & 0
        \end{pmatrix} \begin{pmatrix}
            0 \\
            \begin{pmatrix}
                0 \\ 1
            \end{pmatrix}
        \end{pmatrix} = \frac{1}{\sqrt{E_p + p_z}} \begin{pmatrix}
            ip_y - p_x \\
            E_p + p_z \\
            0 \\
            0
        \end{pmatrix},
    \end{equation*}
    \begin{equation*}
        u_R(p) = \frac{1}{\sqrt{E_p + p_z}} \begin{pmatrix}
            0 & E_p - \vec{p}\cdot\vec{\sigma} \\
            E_p + \vec{p}\cdot\vec{\sigma} & 0
        \end{pmatrix} \begin{pmatrix}
            \begin{pmatrix}
                1 \\ 0
            \end{pmatrix} \\
            0
        \end{pmatrix} = \frac{1}{\sqrt{E_p + p_z}} \begin{pmatrix}
            0 \\
            0 \\
            E_p + p_z \\
            p_x + ip_y
        \end{pmatrix},
    \end{equation*}
    where $E_p = |\vec{p}|$.

    \item By defination,
    \begin{equation*}
        s(p_1, p_2) = \frac{
            E_{p1}(ip_{2y} - p_{2x}) + E_{p2}(p_{1x} - ip_{1y}) + ip_{1z}p_{2y}
         - p_{1z}p_{2x} + p_{1x}p_{2z} - ip_{1y}p_{2z}
        } {\sqrt{(E_{p1} + p_{1z})(E_{p2} + p_{2z})}},
    \end{equation*}
    \begin{equation*}
        t(p_1, p_2) = \frac{
            E_{p1}(p_{2x} + ip_{2y}) - E_{p2}(p_{1x} + ip_{1y}) + p_{1z}p_{2x}
         + ip_{1z}p_{2y} - p_{1x}p_{2z} - ip_{1y}p_{2z}
        } {\sqrt{(E_{p1} + p_{1z})(E_{p2} + p_{2z})}},
    \end{equation*}
    so that
    \begin{equation*}
        s(p_1, p_2) = (t(p_2, p_1))^\ast, \qquad s(p_1, p_2) = -s(p_2, p_1)
    \end{equation*}
    and 
    \begin{align*}
        \left| s(p_1, p_2) \right|^2 & = \frac{
            \left[(E_{p2} + p_{2z})p_{1x} - (E_{p1} + p_{1z})p_{2x}\right]^2 
            + \left[(E_{p1} + p_{1z})p_{2y} - (E_{p2} + p_{2z})p_{1y}\right]^2
        }
        {(E_{p1} + p_{1z})(E_{p2} + p_{2z})} \\
        & = (E_{p1} + p_{1z})(E_{p2} - p_{2z}) + (E_{p2} + p_{2z})(E_{p1} - p_{1z}) - 2(p_{1x}p_{2x} + p_{1y}p_{2y}) \\
        & = 2(E_{p1}E_{p2} - p_{1x}p_{2x} - p_{1y}p_{2y} - p_{1z}p_{2z})\\
        & = 2 p_1 \cdot p_2.
    \end{align*}
\end{problembody}

\problem \textbf{Majorana fermion.} Recall from Eq. (3.40) that one can write a relativistic equation for a massless 2-component fermion field 
that transforms as the upper two components of a Dirac spinor ($\psi_L$). Call such a 2-component field $\chi_a(x)$, $a = 0, 1$.
\begin{problembody}
    \item \label{ex:cp3:majorana_eq} Show that it is possible to write an equation for $\chi(x)$ as a massive field in the following way:
    \begin{equation}\label{equ:cp3:majorana_eq}
        i\bar{\sigma}\cdot\partial\chi - im\sigma^2\chi^\ast = 0.
    \end{equation}
    That is, first, that the equation is relativistically invariant and, second, that it implies the Klein-Gordon equation, $(\partial^2 + m^2)\chi = 0$.
    This form of the fermion mass is called a Majorana mass term.

    \item \label{ex:cp3:majorana_lag} Does the Majorana equation follow from a Lagrangian? The mass term would seem to be the variation of $(\sigma^2)_{ab} \chi_a^\ast \chi_b^\ast$; 
    however, since $\sigma^2$ is antisymmetric, this expression would vanish if $\chi(x)$ were an ordinary c-number field. When we go to quantum field theory,
    we know that $\chi(x)$ will become an anticommuting quantum field. Therefore, it makes sense to develop its classical theory by considering $\chi(x)$ as 
    a classical anticommuting field, that is, as a field that takes as values \textit{Grassmann numbers} which satisfy
    \begin{equation*}
        \alpha \beta = -\beta \alpha \qquad \text{for any} \, \alpha, \beta.
    \end{equation*}
    Note that this relation implies that $\alpha^2 = 0$. A Grassmann field $\xi(x)$ can be expanded in a basis of functions as
    \begin{equation*}
        \xi(x) = \sum_n \alpha_n \phi_n(x),
    \end{equation*}
    where the $\phi_n(x)$ are orthogonal c-number functions and the $\alpha_n$ are a set of independent Grassmann numbers. Define the complex conjugate of a 
    product of Grassmann numbers to reverse the order:
    \begin{equation*}
        (\alpha\beta)^\ast = \beta^\ast \alpha^\ast = -\alpha^\ast \beta^\ast.
    \end{equation*}
    This rule imitates the Hermitian conjugate of quantum fields. Show that the classical action,
    \begin{equation}\label{equ:cp3:majorana_action}
        S = \int \dd^4x \, \left[\chi^\dagger i\bar{\sigma}\cdot\partial\chi + \frac{im}{2}\left(\chi^T\sigma^2\chi - \chi^\dagger\sigma^2\chi^\ast\right)\right],
    \end{equation}
    (where $\chi^\dagger = (\chi^\ast)^T$) is real ($S^\ast = S$), and that varying this $S$ with respect to $\chi$ and $\chi^\ast$ yields the Majorana equation.

    \item \label{ex:cp3:majorana_et_dirac} Let us write a 4-component Dirac field as 
    \begin{equation*}
        \psi(x) = \begin{pmatrix}
            \psi_L \\ \psi_R
        \end{pmatrix},
    \end{equation*}
    and recall that the lower component of $\psi$ transform in a way equivalent by a unitary transformation to the complex conjugate of the representation $\psi_L$. 
    In this way, we can rewrite the 4-component Dirac field in terms of two 2-component spinors:
    \begin{equation*}
        \psi_L(x) = \chi_1(x), \qquad \psi_R(x) = i\sigma^2 \chi^\ast_2(x).
    \end{equation*}
    Rewrite the Dirac Lagrangian in terms of $\chi_1$ and $\chi_2$ and note the form of the mass term.

    \item Show that the action of part \ref{ex:cp3:majorana_et_dirac} has a global symmetry. Compute the divergences of the currents
    \begin{equation*}
        J^\mu = \chi^\dagger \bar{\sigma}^\mu \chi, \qquad J^\mu = \chi_1^\dagger \bar{\sigma}^\mu \chi_1 - \chi_2^\dagger \bar{\sigma}^\mu \chi_2,
    \end{equation*}
    for the theories of parts \ref{ex:cp3:majorana_lag} and \ref{ex:cp3:majorana_et_dirac}, respectively, and relate your results to the symmetries of these theories. 
    Construct a theory of $N$ free massive 2-component fermion fields with $O(N)$ symmetry (that is, the symmetry of rotations in an N-dimensional space).

    \item Quantize the Majorana theory of parts \ref{ex:cp3:majorana_eq} and \ref{ex:cp3:majorana_lag}. That is, promote $\chi(x)$ to a quantum field satisfying the 
    canonical anticommutation relation
    \begin{equation*}
        \left\{\chi_a(\vec{x}), \chi_b^\dagger(\vec{y})\right\} = \delta_{ab}\delta^{(3)}(\vec{x} - \vec{y}),
    \end{equation*}
    construct a Hermitian Hamitonian, and find a representation of the canonical commutation relations that diagonalizes the Hamitonian in terms of a set of creation 
    and annihilation operators. (Hint: Compare $\chi(x)$ to the top two components of the quantized Dirac field.)
\end{problembody}

\solution
\begin{problembody}
    \item Apply an infinitesimal Lorentz transformation $x^\mu \to \Lambda^\mu{}_\nu x^\nu = x^\mu + \omega^\mu{}_\nu x^\nu$, where $\omega_{\mu\nu}$ is defined by \eqref{equ:cp3:omega_def}, then
    $\chi(x)$ should transform as \eqref{equ:cp3:spinor_trans_L}. Writing the transformation of $\chi(x)$ as $\chi \to \Lambda_{\frac{1}{2}}\chi$, \eqref{equ:cp3:majorana_eq} 
    becomes
    \begin{equation*}
        i\bar{\sigma}^\mu \Lambda_{\frac{1}{2}} (\Lambda^{-1})^\nu{}_\mu \partial_\nu\chi - im\sigma^2\Lambda_{\frac{1}{2}}^\ast \chi^\ast = 0.
    \end{equation*}
    Multiplying $\sigma^2(\Lambda_{\frac{1}{2}}^\ast)^{-1}\sigma^2$, it becomes
    \begin{equation*}
        i\sigma^2(\Lambda_{\frac{1}{2}}^\ast)^{-1}\sigma^2\bar{\sigma}^\mu\Lambda_{\frac{1}{2}}(\Lambda^{-1})^\nu{}_\mu\partial_\nu\chi - im\sigma^2\chi^\ast = 0.
    \end{equation*}
    Thus, if $\Lambda_{\frac{1}{2}}$ satisfies
    \begin{equation}\label{equ:cp3:rela_cond}
        \sigma^2(\Lambda_{\frac{1}{2}}^\ast)^{-1}\sigma^2\bar{\sigma}^\mu \Lambda_{\frac{1}{2}} = \Lambda^\mu{}_\nu\bar{\sigma}^\nu,
    \end{equation}
    then \eqref{equ:cp3:majorana_eq} is relativistically invariant. In fact, when $\mu = 0$,
    \begin{align*}
        \sigma^2(\Lambda_{\frac{1}{2}}^\ast)^{-1}\sigma^2\bar{\sigma}^0 \Lambda_{\frac{1}{2}}
        & = \sigma^2\left(1 - i\vec{\theta}\cdot\frac{\vec{\sigma}^T}{2} + \vec{\beta}\cdot\frac{\vec{\sigma}^T}{2}\right)\sigma^2
        \left(1 - i\vec{\theta}\cdot\frac{\vec{\sigma}}{2} - \vec{\beta}\cdot\frac{\vec{\sigma}}{2}\right) \\
        & = \left(1 + i\vec{\theta}\cdot\frac{\vec{\sigma}}{2} - \vec{\beta}\cdot\frac{\vec{\sigma}}{2}\right) 
        \left(1 - i\vec{\theta}\cdot\frac{\vec{\sigma}}{2} - \vec{\beta}\cdot\frac{\vec{\sigma}}{2}\right) \\
        & = 1 - \vec{\beta}\cdot\vec{\sigma}\\
        & = \bar{\sigma}^0 + \omega^0{}_i \bar{\sigma}^i \\
        & = \Lambda^0{}_\nu \bar{\sigma}^\nu,
    \end{align*}
    when $\mu = i$,
    \begin{align*}
        \sigma^2(\Lambda_{\frac{1}{2}}^\ast)^{-1}\sigma^2\bar{\sigma}^i \Lambda_{\frac{1}{2}}
        & = -\left(1 + i\vec{\theta}\cdot\frac{\vec{\sigma}}{2} - \vec{\beta}\cdot\frac{\vec{\sigma}}{2}\right)
        \sigma^i
        \left(1 - i\vec{\theta}\cdot\frac{\vec{\sigma}}{2} - \vec{\beta}\cdot\frac{\vec{\sigma}}{2}\right) \\
        & = -\left(\sigma^i + i\frac{\theta^j}{2}\left[\sigma^j, \sigma^i\right] - \frac{\beta^j}{2}\left\{\sigma^j, \sigma^i\right\}\right)\\
        & = -\left(\sigma^i - \beta^i\bar{\sigma}^0 + \epsilon_{ijk}\theta^j\sigma^k\right)\\
        & = \left(\delta^i_\nu + \omega^i{}_\nu\right) \bar{\sigma}^\nu\\
        & = \Lambda^i{}_\nu \bar{\sigma}^\nu,
    \end{align*}
    so that \eqref{equ:cp3:rela_cond} is satisfied, thus \eqref{equ:cp3:majorana_eq} is relativistically invariant.
    Next we'll prove that \eqref{equ:cp3:majorana_eq} implies Klein-Gordon equation. Take conjugate of \eqref{equ:cp3:majorana_eq}, we get
    \begin{equation*}
        \bar{\sigma}^\ast \cdot \chi^\ast + m\sigma^2\chi = 0.
    \end{equation*}
    Substitute $\chi^\ast = \frac{1}{m}\sigma^2\bar{\sigma}\cdot\partial\chi$, the above becomes
    \begin{equation}\label{equ:cp3:majorana_eq_to_kg}
        \sigma^2 (\bar{\sigma}^\ast \cdot \partial) \sigma^2 (\bar{\sigma} \cdot \partial) \chi + m^2 \chi = 0.
    \end{equation}
    Since 
    \begin{equation*}
        \sigma^2 (\bar{\sigma}^\ast \cdot \partial) \sigma^2 (\bar{\sigma} \cdot \partial) = (\sigma\cdot\partial)(\bar{\sigma}\cdot\partial) = \partial^2,
    \end{equation*}
    so that \eqref{equ:cp3:majorana_eq_to_kg} becomes the Klein-Gordon equation $(\partial^2 + m^2)\chi = 0$.

    \item Writing the spinor indices explicitly in the action, \eqref{equ:cp3:majorana_action} becomes
    \begin{equation*}
        S = \int \dd^4x \, \left[\chi_a^\ast i(\bar{\sigma})_{ab}\cdot\partial\chi_b + \frac{im}{2}\left(\chi_a(\sigma^2)_{ab}\chi_b - \chi_a^\ast (\sigma^2)_{ab}\chi_b^\ast\right)\right].
    \end{equation*}
    The conjugate of this action reads
    \begin{equation*}
        S^\ast = \int \dd^4x \, \left[\chi_a i(\bar{\sigma})_{ba}\cdot\partial\chi_b^\ast - \frac{im}{2}\left(-\chi_b^\ast (\sigma^2)_{ab}\chi_a^\ast + \chi_b (\sigma^2)_{ab} \chi_a\right)\right],
    \end{equation*}
    where we have used $\bar{\sigma}^\ast = \bar{\sigma}^T$, $(\sigma^2)^\ast = -\sigma^2$ and the fact that $\chi$ is Grassmann field. The quadratic terms are unchanged 
    under this conjugate. Integrate by parts, we can see the kinetic term is also invariant:
    \begin{equation*}
        \int \dd^4x \left[\chi_a i(\bar{\sigma})_{ba}\cdot\partial\chi_b^\ast\right] = (\text{surface term}) + \int \dd^4x \, \chi_b^\ast i (\bar{\sigma})_{ba}\cdot\partial\chi_a.
    \end{equation*}
    Thus the action is real. Substitute this action into the Euler-Lagrange equation for $\chi^\ast$:
    \begin{equation*}
        \pd{\lag}{\chi_a^\ast} = \partial_\nu \left(\pd{\lag}{(\partial_\nu\chi_a^\ast)}\right),
    \end{equation*}
    the RHS vanishes. As for the LHS, notice that $\chi$ is Grassmannian, the derivative of the quadratic term is
    \begin{equation*}
        \frac{\partial}{\partial\chi_a^\ast} \left[\chi_b^\ast (\sigma^2)_{bc} \chi_c^\ast\right]
        = (\sigma^2)_{ab} \chi_b^\ast - \chi_b^\ast(\sigma^2)_{ba}
        = 2 (\sigma^2)_{ab} \chi_b^\ast.
    \end{equation*}
    Thus the LHS yields
    \begin{equation*}
        i(\bar{\sigma})_{ab}\cdot\partial\chi_b - im(\sigma^2)_{ab}\chi_b^\ast = 0,
    \end{equation*}
    which is \eqref{equ:cp3:majorana_eq} with spinor indices.

    \item The Lagrangian is
    \begin{align}\label{equ:cp3:dirac_lag}
        \lag = \bar{\psi}\left(i\slashed{\partial} - m\right)\psi 
        & = \left(-i\chi_2^T\sigma^2, \chi_1^\dagger\right) \begin{pmatrix}
            -m & i\sigma\cdot\partial \\
            i\bar{\sigma}\cdot\partial & -m
        \end{pmatrix} \begin{pmatrix}
            \chi_1 \\ i\sigma^2\chi_2^\ast
        \end{pmatrix} \nonumber\\
        & = \chi_1^\dagger i\bar{\sigma}\cdot\partial\chi_1 
        - i\partial\chi_2^\dagger\cdot\bar{\sigma}\chi_2
        - im \left(\chi_1^\dagger\sigma^2\chi_2^\ast - \chi_2^T\sigma^2\chi_1\right).
    \end{align}
    The second term can be written in a similiar form as the first term by integrating by parts the action:
    \begin{equation*}
        \int \dd^4 x \, i\partial\chi_2^\dagger\cdot\bar{\sigma}\chi_2 = (\text{surface term}) - \int \dd^4x \, \chi_2^\dagger i\bar{\sigma}\cdot\partial\chi_2.
    \end{equation*}
    So \eqref{equ:cp3:dirac_lag} can be written equivalently as
    \begin{equation*}
        \lag = \chi_1^\dagger i\bar{\sigma}\cdot\partial\chi_1
         + \chi_2^\dagger i\bar{\sigma}\cdot\partial\chi_2
         - im \left(\chi_1^\dagger\sigma^2\chi_2^\ast - \chi_2^T\sigma^2\chi_1\right).
    \end{equation*}
    The Dirac equation can be written in terms of $\chi_1$ and $\chi_2$ as
    \begin{equation*}
        (i\slashed{\partial} - m)\psi = \begin{pmatrix}
            -m & i\sigma\cdot\partial \\
            i\bar{\sigma}\cdot\partial & -m
        \end{pmatrix} \begin{pmatrix}
            \chi_1 \\ i\sigma^2 \chi_2^\ast
        \end{pmatrix} = \begin{pmatrix}
            -(\sigma\cdot\partial)\sigma^2\chi_2^\ast - m\chi_1 \\
            i\bar{\sigma}\cdot\partial\chi_1 - im\sigma^2\chi_2^\ast
        \end{pmatrix} = 0,
    \end{equation*}
    that is
    \begin{subequations}\label{equ:cp3:dirac_eq_2}
        \begin{align}
            i\bar{\sigma}\cdot\partial\chi_1 - im\sigma^2\chi_2^\ast & = 0, \\
            i\bar{\sigma}\cdot\partial\chi_2 - im\sigma^2\chi_1^\ast & = 0,
        \end{align}
    \end{subequations}
    compared to \eqref{equ:cp3:majorana_eq}.

    \item Since the original Dirac action is invariant under global symmetry $\psi \to e^{-i\alpha}\psi$, \eqref{equ:cp3:dirac_lag} then also has a global symmetry
    \begin{subequations}\label{equ:cp3:dirac_sym2}
        \begin{align}
            \chi_1 & \to e^{-i\alpha}\chi_1, \\
            \chi_2 & \to e^{i\alpha}\chi_2.
        \end{align}
    \end{subequations}
    For the first current,
    \begin{equation*}
        \partial_\mu J^\mu = \partial\chi^\dagger\cdot\bar{\sigma} \chi + \chi^\dagger\bar{\sigma}\cdot\partial\chi
        = m\chi^T\sigma^2\chi + m\chi^\dagger\sigma^2\chi^\ast,
    \end{equation*}
    which does not vanish, so this current is not conserved\needverify.
    For the second current,
    \begin{align*}
        \partial_\mu J^\mu & = \partial\chi_1^\dagger\cdot\bar{\sigma}\chi_1 + \chi_1^\dagger\bar{\sigma}\cdot\partial\chi_1
        - \partial\chi_2^\dagger\cdot\bar{\sigma}\chi_2 - \chi_2^\dagger\bar{\sigma}\cdot\partial\chi_2 \\
        & = m\chi_2^T\sigma^2\chi_1 + m\chi_1^\dagger\sigma^2\chi_2^\ast - m\chi_1^T\sigma^2\chi_2 - m\chi_2^\dagger\sigma^2\chi_1^\ast\\
        & = 0,
    \end{align*}
    which is an expected result, since it is the corresponding Neother current of symmetry \eqref{equ:cp3:dirac_sym2}.
    The action of the theory with N massive 2-component fermion fields can be obtained simply by adding the action of each field:
    \begin{equation*}
        S = \sum_i \int \dd^4x \, \left[\chi_i^\dagger i\bar{\sigma}\cdot\partial\chi_i + \frac{im}{2}\left(\chi_i^T\sigma^2\chi_i - \chi_i^\dagger\sigma^2\chi_i^\ast\right)\right].
    \end{equation*}
    This action is invariant under the global transformation
    \begin{equation*}
        \chi_i \to R_{ij}\chi_j,
    \end{equation*}
    where $R \in O(N)$ is an $N \times N$ real orthogonal matrix.

    \item First, expand $\chi$ in terms of plan wave functions
    \begin{equation}\label{equ:cp3:majorana_expand}
        \chi(x) = \sum_k \left(u_k \varphi_k + v_k \varphi_k^\ast\right),
    \end{equation}
    where $\varphi_k$ is defined by \eqref{equ:cp2:plain_w}. Substitute into \eqref{equ:cp3:majorana_eq}, we get
    \begin{subequations}
        \begin{align}
            k\cdot\bar{\sigma} u_k & = im\sigma^2 v_k^\ast, \\
            k\cdot\bar{\sigma} v_k & = -im\sigma^2 u_k^\ast.
        \end{align}
    \end{subequations}
    The second equation is automatically satisfied if $k^2 = m^2$. Then the solution can be written as
    \begin{subequations}\label{equ:cp3:majorana_spinor}
        \begin{align}
            u_k & = \sqrt{k\cdot\sigma}\xi a_k,\\
            v_k & = \sqrt{k\cdot\sigma} (-i\sigma^2) \xi^\ast a_k^\ast,
        \end{align}
    \end{subequations}
    where $a_k$ are Grassmann variables depend on $k$, and $\xi$ a 2-component variable, related to the particle's spin. Now  Choosing a basis $\{\xi^1, \xi^2\}$ of $\xi$ just
    as we've done with the Dirac spinors, and redefine $u_k$ and $v_k$ as
    \begin{subequations}
        \begin{align}
            u_k^s & = \sqrt{k\cdot\sigma} \xi^s, \\
            v_k^s & = \sqrt{k\cdot\sigma} (-i\sigma^2) \xi^{s\ast}.
        \end{align}
    \end{subequations}
    The orthogonal relations of $u_k^s$ and $v_k^s$ read
    \begin{subequations}
        \begin{equation}
            (v_k^s)^T\sigma^2 u_k^{s'} = -(u_k^{s'})^T\sigma^2 v_k^s = im\delta_{ss'},
        \end{equation}
        \begin{equation}
            (v_k^s)^H\sigma^2 (u_k^{s'})^\ast = -(u_k^{s'})^H\sigma^2 (v_k^s)^\ast = -im\delta_{ss'}.
        \end{equation}
    \end{subequations}
    To quantize $\chi$, promoting $a_k$ to operators. To avoid ambiguity we will use ${}^\dagger$ and ${}^H$ to denote Hermitian conjugations of
    operators and spinors respectively. We may start by guessing that \eqref{equ:cp3:majorana_expand} can be written as
    \begin{equation}\label{equ:cp3:majorana_expand_q}
        \chi(x) = \sum_{k, s} \left(a_k^s \varphi_k^s + a_k^{s\dagger} \bar{\varphi}_k^s\right),
    \end{equation}
    where
    \begin{subequations}
        \begin{align}
            \varphi_k^s & = u_k^s \varphi_k = \sqrt{k\cdot\sigma}\xi^s \varphi_k,\\
            \bar{\varphi}_k^s & = v_k^s \varphi_k^\ast = \sqrt{k\cdot\sigma} (-i\sigma^2) \xi^{s\ast} \varphi_k^\ast,
        \end{align}
    \end{subequations}
    and $a_k^s$ satisfies the anticommutation relation $\{a_k^s, a_{k'}^{s'}\} = \delta_{ss'}\delta_{kk'}$. We can now verify that $\chi$ and $\chi^\dagger$ satisfy the canonical anticommutation relation: 
    \begin{align*}
        \{\chi_a(\vec{x}), \chi_b^\dagger(\vec{y})\} & = \sum_{k, s}\sum_{k', s'}\left\{a_k^s(\varphi_k^s)_a + a_k^{s\dagger}(\bar{\varphi}_k^s)_a, a_{k'}^{s'\dagger}(\varphi_{k'}^{s'\ast})_b + a_{k'}^{s'}(\bar{\varphi}_{k'}^{s'\ast})_b\right\}\\
        & = \sum_{k, s}\left(\varphi_k^s(\vec{x})\varphi_k^{sH}(\vec{y}) + \bar{\varphi}_k^s(\vec{x})\bar{\varphi}_k^{sH}(\vec{y})\right)_{ab}\\
        & = \delta_{ab} \frac{1}{V}\sum_k\frac{1}{2\omega_k}\left(k\cdot\sigma e^{i\vec{k}\cdot(\vec{x} - \vec{y})} + k\cdot\sigma e^{-i\vec{k}\cdot(\vec{x} - \vec{y})}\right)\\
        & = \delta_{ab} \frac{1}{V}\sum_k e^{i\vec{k}\cdot(\vec{x} - \vec{y})}\\
        & = \delta_{ab}\delta^3(\vec{x} - \vec{y}).
    \end{align*}
    The canonical conjugate momentum of $\chi$ is
    \begin{equation*}
        \pi_a = \pd{\lag}{\dot{\chi_a}} = -i\chi_a^\dagger.
    \end{equation*}
    Now construct the Hamitonian. To do this, we could use the defination $H = \sum_i p_i\dot{x_i} - L$. However, since we are working with Grassmann fields, the order of $p$ and $\dot{x}$ in the first term is not clear. But the
    Hamitonian should reproduce \eqref{equ:cp3:majorana_eq} when substitute into Heisenburg equation, from this point we can construct the Hamitonian as
    \begin{equation}\label{equ:cp3:hamitonian_marjorana}
        H = \int \dd^3x \, \left[
            i\chi^{T\dagger} \vec{\sigma}\cdot\nabla\chi 
            - \frac{im}{2}\left(
                \chi^T\sigma^2\chi 
                - \chi^{T\dagger}\sigma^2\chi^\dagger
            \right)
        \right].
    \end{equation}
    Now substitute \eqref{equ:cp3:majorana_expand_q} into \eqref{equ:cp3:hamitonian_marjorana}:
    \begin{align*}
        H & = \sum_{k, s}\sum_{k', s'} \int \dd^3x \left[
            -\left(a_{k'}^{s'\dagger}\varphi_{k'}^{s'H} + a_{k'}^{s'}\bar{\varphi}_{k'}^{s'H}\right)
            \vec{k}\cdot\vec{\sigma}
            \left(a_k^s\varphi_k^s - a_k^{s\dagger}\bar{\varphi}_k^s\right)
            \right.\\
            &\qquad \left. -\frac{im}{2} 
            \left(a_{k'}^{s'}\varphi_{k'}^{s'T} + a_{k'}^{s'\dagger}\bar{\varphi}_{k'}^{s'T}\right)
            \sigma^2
            \left(a_k^s\varphi_k^s + a_k^{s\dagger}\bar{\varphi}_k^s\right) + \hc
        \right]\\
        & = \sum_{k, s} \omega_k \left(a_k^{s\dagger}a_k^s - \frac{1}{2}\right),
    \end{align*}
    which is a similiar result as that of Dirac field.
\end{problembody}

\problem \textbf{Supersymmetry.} It is possible to write field theories with continous symmetries linking fermions and bosons; such 
transformations are called \textit{supersymmetries}.
\begin{problembody}
    \item\label{ex:cp3:susy1} The simplest example of supersymmetric field theory is the theory of a free complex boson and a free weyl fermion, written in the form
    \begin{equation}\label{equ:cp3:susy_lag}
        \lag = \partial_\mu\phi^\ast\partial^\mu\phi 
            + \chi^\dagger i\bar{\sigma}\cdot\partial\chi
            + F^\ast F.
    \end{equation}
    Here $F$ is an auxiliary complex scalar field whose field equation is $F = 0$. Show that this Lagrangian is invariant (up to a total divergence)
    under the infinitesimal transformation
    \begin{subequations}\label{equ:cp3:susy_sym_trans}
        \begin{align}
            \delta\phi & = -i\epsilon^T\sigma^2\chi,\\
            \delta\chi & = \epsilon F + \sigma\cdot\partial\phi \sigma^2\epsilon^\ast,\\
            \delta F & = -i\epsilon^\dagger\bar{\sigma}\cdot\partial\chi,
        \end{align}
    \end{subequations}
    where the parameter $\epsilon_a$ is a 2-component spinor of Grassmann numbers.

    \item Show that the term
    \begin{equation}\label{equ:cp3:susy_mass_terms}
        \Delta\lag = \left[m\phi F + \frac{1}{2}im\chi^T\sigma^2\chi\right] + (\text{complex conjugate})
    \end{equation}
    is also left invariant by the transformation given in part \ref{ex:cp3:susy1}. Eliminate $F$ from the complete Lagrangian $\lag + \Delta\lag$ by solving its field
    equation, and show that the fermion and boson fields $\phi$ and $\chi$ are given the same mass.

    \item It is possible to write supersymmetric nonlinear field equations by adding cubic and higher-order terms to the Lagrangian. Show that the following rather 
    general field theory, containing the field $(\phi_i, \chi_i), i = 1, \dots, n$, is supersymmetric:
    \begin{align}\label{equ:cp3:susy_lag_g}
        \lag & = \partial_\mu\phi_i^\ast\partial^\mu\phi_i
            + \chi_i^\dagger i\bar{\sigma}\cdot\partial\chi_i + F_i^\ast F_i\nonumber\\
            & \qquad + \left(
                F_i\pd{W[\phi]}{\phi_i} 
                + \frac{i}{2}\pdd{W[\phi]}{\phi_i}{\phi_j}\chi_i^T\sigma^2\chi_j
                + \cc
            \right),
    \end{align}
    where $W[\phi]$ is an arbitary function of the $\phi_i$, called the \textit{superpotential}. For the simplest case $n = 1$ and $W = g\phi^3 / 3$, write out the field 
    equations for $\phi$ and $\chi$ (after elimination of $F$).
\end{problembody}

\solution
\begin{problembody}
    \item Substitute the transformation \eqref{equ:cp3:susy_sym_trans} into \eqref{equ:cp3:susy_lag}, the new Lagrangian is 
    \begin{align*}
        \lag' & = \left(\partial_\mu\phi^\ast + i\partial_\mu\chi^\dagger\sigma^2\epsilon^\ast\right)
        \left(\partial^\mu\phi - i\epsilon^T\sigma^2\partial^\mu\chi\right)\\
        & \qquad + \left(\chi^\dagger + \epsilon^\dagger F^\ast + \epsilon^T\sigma^2\sigma\cdot\partial\phi^\ast \right)
        \left(i\bar{\sigma}\cdot\partial\chi + i\bar{\sigma}\cdot\partial F\epsilon + i\sigma^2\epsilon^\ast\partial^2\phi\right)\\
        & \qquad + \left(F^\ast + i\partial\chi^\dagger\cdot\bar{\sigma}\epsilon\right)
        \left(F - i\epsilon^\dagger\bar{\sigma}\cdot\partial\chi\right)\\
        %
        & = \lag + i\partial^\mu\phi\partial_\mu\chi^\dagger\sigma^2\epsilon^\ast
        - i\partial_\mu\phi^\ast\epsilon^T\sigma^2\partial^\mu\chi
        + \chi^\dagger i\bar{\sigma}\cdot\partial F\epsilon
        + \chi^\dagger i\sigma^2\epsilon^\ast\partial^2\phi\\
        & \qquad + \underline{iF^\ast\epsilon^\dagger\bar{\sigma}\cdot\partial\chi}
        + i\epsilon^T\sigma^2\sigma\cdot\partial\phi^\ast\bar{\sigma}\cdot\partial\chi
        + iF\partial\chi^\dagger\cdot\bar{\sigma}\epsilon
        - \underline{iF^\ast\epsilon^\dagger\bar{\sigma}\cdot\partial\chi}\\
        & \qquad + o(\epsilon)\\
        %
        & = \lag + \partial_\mu \left[
            \chi^\dagger i\sigma^2\epsilon^\ast\partial^\mu\phi
            + i\phi^\ast\epsilon^T\sigma^2 \left(
                \sigma^\mu\bar{\sigma}\cdot\partial\chi
                - \partial^\mu\chi
            \right)
            + F\chi^\dagger i\bar{\sigma}^\mu\epsilon
        \right] + o(\epsilon).
    \end{align*}
    So that the Lagrangian is invariant up to a divergence.

    \item Under the transformation \eqref{equ:cp3:susy_sym_trans}, \eqref{equ:cp3:susy_mass_terms} transform as
    \begin{align*}
        \Delta\lag & \to m\left(\phi - i\epsilon^T\sigma^2\chi\right)
        \left(F - i\epsilon^\dagger\bar{\sigma}\cdot\partial\chi\right)\\
        & \qquad + \frac{1}{2}im
        \left(
            \chi^T 
            + F\epsilon^T 
            - \epsilon^\dagger\sigma^2\sigma^\ast\cdot\partial\phi
        \right)
        \sigma^2
        \left(
            \chi 
            + \epsilon F 
            + \sigma\cdot\partial\phi\sigma^2\epsilon^\ast
        \right)
        + \cc\\
        %
        & = \Delta\lag + \left(
            - im\phi\epsilon^\dagger\bar{\sigma}\cdot\partial\chi
            - imF\epsilon^T\sigma^2\chi
            + \frac{1}{2}imF\chi^T\sigma^2\epsilon
            + \frac{1}{2}im\chi^T\bar{\sigma}^\ast\cdot\partial\phi\epsilon^\ast\right.\\
            & \qquad \left. + \frac{1}{2}imF\epsilon^T\sigma^2\chi
            - \frac{1}{2}im\epsilon^\dagger\bar{\sigma}\cdot\partial\phi\chi
            + \cc
        \right) + o(\epsilon)\\
        %
        & = \Delta\lag + \partial_\mu\left(
            -im\phi\epsilon^\dagger\bar{\sigma}^\mu\chi + \cc
        \right) + o(\epsilon),
    \end{align*}
    which is invariant up to a total divergence. The full Lagrangian now reads
    \begin{equation}\label{equ:cp3:susy_massive_lag}
        \lag = \partial_\mu\phi^\ast\partial^\mu\phi
        + m (\phi F + \phi^\ast F^\ast)
        + \chi^\dagger i\bar{\sigma}\cdot\partial\chi
        + \frac{im}{2}\left(\chi^T\sigma^2\chi - \chi^\dagger\sigma^2\chi\right)
        + F^\ast F,
    \end{equation}
    notice that the fermion part is just \eqref{equ:cp3:majorana_action}, with the mass being $m$. Varying $F^\ast$, we get the equation of motion for $F$:
    \begin{equation*}
        F + m\phi^\ast = 0.
    \end{equation*}
    Thus \eqref{equ:cp3:susy_massive_lag} becomes
    \begin{equation*}
        \lag = \partial_\mu\phi^\ast\partial^\mu\phi
        - m^2\phi^\ast\phi
        + \chi^\dagger i\bar{\sigma}\cdot\partial\chi
        + \frac{im}{2}\left(\chi^T\sigma^2\chi - \chi^\dagger\sigma^2\chi\right).
    \end{equation*}

    \item The transformation \eqref{equ:cp3:susy_sym_trans} can be generalized as
    \begin{subequations}\label{equ:cp3:susy_trans_g}
        \begin{align}
            \delta\phi_i & = -i\epsilon^T\sigma^2\chi_i,\\
            \delta\chi_i & = \epsilon F_i + \sigma\cdot\partial\phi_i\sigma^2\epsilon^\ast,\\
            \delta F_i & = -i\epsilon^\dagger\bar{\sigma}\cdot\partial\chi_i.
        \end{align}
    \end{subequations}
    The first three terms of \eqref{equ:cp3:susy_lag_g} is left invariant, while the rest transforms as
    \begin{align*}
        \lag_W & \to \left(
            F_i - i\epsilon^\dagger\bar{\sigma}\cdot\partial\chi_i
        \right)
        \left(
            \pd{W[\phi]}{\phi_i} 
            - \pdd{W[\phi]}{\phi_i}{\phi_j}i\epsilon^T\sigma^2\chi_j
        \right)\\
        & \qquad + i\pdd{W[\phi]}{\phi_i}{\phi_j}
        \left(
            \chi_i^T\sigma^2\chi_j
            + F_i\epsilon^T\sigma^2\chi_j
            - \epsilon^\dagger\bar{\sigma}\cdot\partial\phi_i\chi_j    
        \right) + \cc\\
        %
        & = \lag_W + \partial_\mu\left(
            -i\epsilon^\dagger\bar{\sigma}^\mu\chi_i\pd{W[\phi]}{\phi_i} + \cc    
        \right) + o(\epsilon),
    \end{align*}
    so that \eqref{equ:cp3:susy_lag_g} is invariant under \eqref{equ:cp3:susy_trans_g}. Equation of motion for $F_i$
    can be obtained by varying $F_i^\ast$ in \eqref{equ:cp3:susy_lag_g}:
    \begin{equation*}
        F_i + \left(\pd{W[\phi]}{\phi_i}\right)^\ast = 0.
    \end{equation*}
    Substitute $F_i$ and $F_i^\ast$ back to \eqref{equ:cp3:susy_lag_g}, we get
    \begin{equation*}
        \lag = \partial_\mu\phi_i^\ast\partial^\mu\phi_i
        - \left|\pd{W}{\phi_i}\right|^2
        + \chi_i^\dagger i\bar{\sigma}\cdot\partial\chi_i
        + \frac{i}{2}\left(
            \pdd{W}{\phi_i}{\phi_j}\chi_i^T\sigma^2\chi_j
            - \left(\pdd{W}{\phi_i}{\phi_j}\right)^\ast\chi_i^\dagger\sigma^2\chi_j^\ast
        \right).
    \end{equation*}
    Thus, the field equations of $\phi_i$ and $\chi_i$ are
    \begin{equation*}
        \left\{
        \begin{aligned}
            & \partial^2\phi_i + \left(\pd{W}{\phi_j}\right)^\ast
            \pdd{W}{\phi_j}{\phi_i} = 0,\\
            & i\bar{\sigma}\cdot\partial\chi_i
            - i\left(\pdd{W}{\phi_i}{\phi_j}\right)^\ast\sigma^2\chi_j^\ast = 0.
        \end{aligned}
        \right.
    \end{equation*}
\end{problembody}

\problem \textbf{Fierz transformations.} Let $u_i$, $i = 1, \dots, 4$, be four 4-component Dirac spinors.
In the text, we proved the Fierz rearrangement formulae (3.78) and (3.79). The first of these formulae can
be written in 4-component notation as
\begin{equation*}
    \bar{u}_1\gamma^\mu\left(\frac{1 + \gamma^5}{2}\right)u_2
    \bar{u}_3\gamma_\mu\left(\frac{1 + \gamma^5}{2}\right)u_4 = -
    \bar{u}_1\gamma^\mu\left(\frac{1 + \gamma^5}{2}\right)u_4
    \bar{u}_3\gamma_\mu\left(\frac{1 + \gamma^5}{2}\right)u_2.
\end{equation*}
In fact, therer are similar rearrangement formulae for any product
\[
    (\bar{u}_1\Gamma^Au_2)(\bar{u}_3\Gamma^Bu_4),
\]
where $\Gamma^A$, $\Gamma^B$ are any of the 16 combinations of Dirac matrices listed in Section 3.4.
\begin{problembody}
    \item To begin, normalize the 16 matrices $\Gamma^A$ to the convention
    \begin{equation}\label{equ:cp3:gamma_convention}
        \tr\left[\Gamma^A\Gamma^B\right] = 4\delta^{AB}.
    \end{equation}
    This gives $\Gamma^A = \{1, \gamma^0, i\gamma^j, \dots\}$; write all 16 elements of this set.

    \item Write the general Fierz identity as an equation
    \begin{equation}\label{equ:cp3:fierz_trans_g}
        (\bar{u}_1\Gamma^Au_2)(\bar{u}_3\Gamma^Bu_4) =
        \sum_{C, D} C^{AB}{}_{CD} (\bar{u}_1\Gamma^Cu_4)(\bar{u}_3\Gamma^Du_2),
    \end{equation}
    with unknown coefficients $C^{AB}{}_{CD}$. Using the completeness of the 16 $\Gamma^A$ matrices, show that
    \begin{equation*}
        C^{AB}{}_{CD} = \frac{1}{16}\tr\left[\Gamma^C\Gamma^A\Gamma^D\Gamma^B\right].
    \end{equation*}

    \item Work out explicitly the Fierz transformation laws for the products $(\bar{u}_1u_2)(\bar{u}_3u_4)$ and 
    $(\bar{u}_1\gamma^\mu u_2)(\bar{u}_3\gamma_\mu u_4)$.
\end{problembody}

\solution
\begin{problembody}
    \item To do this, calculate the traces of the products among these 5 sets of matrices. The non-zero traces are:
    \begin{align*}
        \tr[1 1] & = 4,\\
        \tr[\gamma^\mu\gamma^\nu] & = 4g^{\mu\nu},\\
        \tr[\sigma^{\mu\nu}\sigma^{\alpha\beta}] 
        & = -\tr[\gamma^{[\mu}\gamma^{\nu]}\gamma^{[\alpha}\gamma^{\beta]}]
        = 4(g^{\mu\beta}g^{\nu\alpha} - g^{\mu\alpha}g^{\nu\beta}),\\
        \tr[\gamma^\mu\gamma^5\gamma^\nu\gamma^5] 
        & = -\tr[\gamma^\mu\gamma^\nu]
        = -4g^{\mu\nu},\\
        \tr[\gamma^5\gamma^5] & = 4.
    \end{align*}
    The other all vanish, since the trace of any odd number of $\gamma^\mu$ is zero. 
    In order to maintain \eqref{equ:cp3:gamma_convention}, the elements of $\Gamma^A$ can be chosen as
    \begin{equation*}
        \Gamma^A = \{
            1, \gamma^0, i\gamma^i, \sigma^{0i}, i\sigma^{ij}, 
            i\gamma^0\gamma^5, \gamma^i\gamma^5, \gamma^5
        \}.
    \end{equation*}

    \item \eqref{equ:cp3:fierz_trans_g} is satisfied if the following formula is true:
    \begin{equation*}
        (\Gamma^A)_{ab}(\Gamma^B)_{cd}
        = \sum_{C', D'} C^{AB}{}_{C'D'}(\Gamma^{C'})_{ad}(\Gamma^{D'})_{cb},
    \end{equation*}
    where $a, b, c, d$ are Dirac indices.
    Multiplying $(\Gamma^C)_{da}(\Gamma^D)_{bc}$ on both sides and sum over all Dirac indices, RHS becomes two
    traces of the form \eqref{equ:cp3:gamma_convention}, exposing $C^{AB}{}_{CD}$:
    \begin{equation*}
        C^{AB}{}_{CD} = \frac{1}{16}(\Gamma^A)_{ab}(\Gamma^B)_{cd}(\Gamma^C)_{da}(\Gamma^D)_{bc}
        = \frac{1}{16}\tr[\Gamma^C\Gamma^A\Gamma^D\Gamma^B].
    \end{equation*}
    Notice that we haven't used the fact that the $u$'s are Dirac spinors, they are only required to be $1\times 4$ matrices.
    Also, we have only used the completeness relations of the two matrices on the RHS, while the two on the LHS can actually
    be any $4\times 4$ matrix. Thus \eqref{equ:cp3:fierz_trans_g} can be generalized to
    \begin{equation}\label{equ:cp3:fierz_trans_2}
        (\bar{u}_1 \Gamma_1 u_2)(\bar{u}_3 \Gamma_2 u_4) =
        \sum_{A, B} C_{AB} (\bar{u}_1\Gamma^Au_4)(\bar{u}_3\Gamma^Au_2),
    \end{equation}
    with $C_{AB} = \frac{1}{16}\tr[\Gamma^C\Gamma_1\Gamma^D\Gamma_2]$ and $\Gamma_1$ and $\Gamma_2$ are two matrices.

    \item For the first product, let $\Gamma^A = \Gamma^B = 1$, then $C^{AB}{}_{CD} = \frac{1}{16}\tr[\Gamma^C\Gamma^D] = \frac{1}{4}\delta^{CD}$.
    Thus
    \begin{align*}
        (\bar{u}_1u_2)(\bar{u}_3u_4) & = \frac{1}{4}\sum_C (\bar{u}_1\Gamma^Cu_4)(\bar{u}_3\Gamma^Cu_2)\\
        & = \frac{1}{4}\left[
            (\bar{u}_1u_4)(\bar{u}_3u_2) 
            + (\bar{u}_1\gamma^\mu u_4)(\bar{u}_3\gamma_\mu u_2)
            - \frac{1}{2}(\bar{u}_1\sigma^{\mu\nu}u_4)(\bar{u}_3\sigma_{\mu\nu}u_2)\right.\\
            & \qquad \left.- (\bar{u}_1\gamma^\mu\gamma^5u_4)(\bar{u}_3\gamma_\mu\gamma^5u_2)
            + (\bar{u}_1\gamma^5u_4)(\bar{u}_3\gamma^5u_2)
        \right].
    \end{align*}
    For the second one, set $\Gamma_1 = \gamma^\mu$ and $\Gamma_2 = \gamma_\mu$ in \eqref{equ:cp3:fierz_trans_2}, the coefficients are
    \begin{equation*}
        C_{AB} = \frac{1}{16}\tr[\Gamma^A\gamma^\mu\Gamma^B\gamma_\mu].
    \end{equation*}
    The non-zero traces related to $C_{AB}$ are:
    \begin{align*}
        \tr[1\gamma^\mu 1\gamma_\mu] & = 4,\\
        \tr[\gamma^\alpha\gamma^\mu\gamma^\beta\gamma_\mu] & = -8g^{\alpha\beta},\\
        % \tr[\sigma^{\alpha\beta}\gamma^\mu\sigma^{\nu\rho}\gamma_\mu] 
        % & = -8\tr[\sigma^{\alpha\beta}\sigma^{\nu\rho}]
        % = 32(g^{\alpha\nu}g^{\beta\rho} - g^{\alpha\rho}g^{\beta\nu}),\\
        \tr[(\gamma^\alpha\gamma^5)\gamma^\mu(\gamma^\beta\gamma^5)\gamma_\mu]
        & = -8g^{\alpha\beta},\\
        \tr[\gamma^5\gamma^\mu\gamma^5\gamma_\mu] & = -4.
    \end{align*}
    Express these in terms of elements in $\{\Gamma^A\}$:
    \begin{align*}
        \tr[1\gamma^\mu 1\gamma_\mu] & = 4,\\
        \tr[\gamma^0\gamma^\mu\gamma^0\gamma_\mu] & = -8,\\
        \tr[(i\gamma^i)\gamma^\mu(i\gamma^i)\gamma_\mu] & = -8,\\
        % \tr[\sigma^{0i}\gamma^\mu\sigma^{0i}\gamma_\mu] & = -32,\\
        % \tr[(i\sigma^{ij})\sigma^\mu(i\sigma^{ij})\gamma_\mu] & = -32,\\
        \tr[(i\gamma^0\gamma^5)\gamma^\mu(i\gamma^0\gamma^5)\gamma_\mu] & = 8,\\
        \tr[(\gamma^i\gamma^5)\gamma^\mu(\gamma^i\gamma^5)\gamma_\mu] & = 8,\\
        \tr[\gamma^5\gamma^\mu\gamma^5\gamma_\mu] & = -4.
    \end{align*}
    Thus
    \begin{align*}
        (\bar{u}_1\gamma^\mu u_2)(\bar{u}_3\gamma_\mu u_4) & = 
        \frac{1}{4}(\bar{u}_1u_4)(\bar{u}_3u_2)
        - \frac{1}{2}(\bar{u}_1\gamma^\mu u_4)(\bar{u}_3\gamma_\mu u_2)\\
        & \qquad - \frac{1}{2}(\bar{u}_1\gamma^\mu\gamma^5u_4)(\bar{u}_3\gamma_\mu\gamma^5u_2)
        - \frac{1}{4}(\bar{u}_1\gamma^5u_4)(\bar{u}_3\gamma^5u_2).
    \end{align*}
\end{problembody}

\problem This problem concerns the dsicrete symmetries \textit{P}, \textit{C}, and \textit{T}.
\begin{problembody}
    \item Compute the transformation properties under $P$, $C$, and $T$ of the antisymmetric
    tensor fermion bilinears, $\bar{\psi}\sigma^{\mu\nu}\psi$, with $\sigma^{\mu\nu} = \frac{i}{2}[\gamma^\mu, \gamma^\nu]$.
    This completes the table of the transformation properties of bilinears at the end of the chapter.

    \item Let $\phi(x)$ be a complex-valued Klein-Gordon field, such as we considered in Problem 2.2.
    Find unitary operators $P$, $C$ and an antiunitary operator $T$ (all defined in terms of their action
    on the annihilation operators $a_p$ and $b_p$ for the Klein-Gordon particles and antiparticles) that
    give the following transformations of the Klein-Gordon field:
    \begin{align*}
        P\phi(t, \vec{x})P & = \phi(t, -\vec{x});\\
        T\phi(t, \vec{x})T & = \phi(-t, \vec{x});\\
        C\phi(t, \vec{x})C & = \phi^\ast(t, \vec{x}).
    \end{align*}
    Find the transformation properties of the components of the current
    \begin{equation*}
        J^\mu = i(\phi^\ast\partial^\mu\phi - \partial^\mu\phi^\ast\phi)
    \end{equation*}
    under $P$, $C$, and $T$.

    \item Show that any Hermitian Lorentz scalar local operator built from $\psi(x)$, $\phi(x)$, and their conjugates
    has $CPT = +1$.
\end{problembody}

\solution
\begin{problembody}
    \item 
    \begin{align*}
        P\bar{\psi}\sigma^{\mu\nu}\psi P
        = i\bar{\psi}\gamma^0\gamma^{[\mu}\gamma^{\nu]}\gamma^0\psi
        = \begin{cases}
            -\bar{\psi}\sigma^{\mu\nu}\psi \qquad & (\mu = 0, \nu = i)\\
            +\bar{\psi}\sigma^{\mu\nu}\psi \qquad & (\mu = i, \nu = j)
        \end{cases}.
    \end{align*}
    For time reversal, when $\mu = 0, \nu = i$,
    \begin{align*}
        T\bar{\psi}\sigma^{0i}\psi T
        = \bar{\psi}\gamma^3\gamma^1(\sigma^{0i})^\ast\gamma^1\gamma^3\psi
        = -i\bar{\psi}\gamma^3\gamma^1\gamma^2\gamma^{[0}\gamma^{i]}\gamma^2\gamma^1\gamma^3\psi
        = \bar{\psi}\sigma^{0i}\psi,
    \end{align*}
    when $\mu = i, \nu = j$,
    \begin{equation*}
        T\bar{\psi}\sigma^{ij}\psi T
        = -i\bar{\psi}\gamma^3\gamma^1\gamma^2\gamma^{[i}\gamma^{j]}\gamma^2\gamma^1\gamma^3\psi
        = -\bar{\psi}\sigma^{ij}\psi,
    \end{equation*}
    where we have used $\gamma^{i\ast} = -\gamma^2\gamma^i\gamma^2$. Thus
    \begin{equation*}
        T\bar{\psi}\sigma^{\mu\nu}\psi T = \begin{cases}
            +\bar{\psi}\sigma^{\mu\nu}\psi \qquad (\mu = 0, \nu = i)\\
            -\bar{\psi}\sigma^{\mu\nu}\psi \qquad (\mu = i, \nu = j)
        \end{cases}.
    \end{equation*}
    As for charge conjugation,
    \begin{align*}
        C\bar{\psi}\sigma^{\mu\nu}\psi C
        = -(\gamma^0\gamma^2\psi)^T\sigma^{\mu\nu}(\bar{\psi}\gamma^0\gamma^2)^T
        = \bar{\psi}\gamma^0\gamma^2(\sigma^{\mu\nu})^T\gamma^0\gamma^2\psi.
    \end{align*}
    When $\mu = 0, \nu = i$:
    \begin{equation*}
        C\bar{\psi}\sigma^{0i}\psi C
        = -i\bar{\psi}\gamma^0\gamma^2\gamma^2\gamma^{[i}\gamma^{|2|}\gamma^{0]}\gamma^0\gamma^2\psi
        = -\bar{\psi}\sigma^{0i}\psi,
    \end{equation*}
    when $\mu = i, \nu = j$:
    \begin{equation*}
        C\bar{\psi}\sigma^{ij}\psi C
        = i\bar{\psi}\gamma^0\gamma^2\gamma^2\gamma^{[j}\gamma^{i]}\gamma^2\gamma^0\gamma^2\psi
        = -\bar{\psi}\sigma^{ij}\psi.
    \end{equation*}
    Thus 
    \[
        C\bar{\psi}\sigma^{\mu\nu}\psi C
        = -\bar{\psi}\sigma^{\mu\nu}\psi.
    \]

    \item As \eqref{equ:cp2:phi_expand}, expand $\phi(x)$ as
    \[
        \phi(x) = \sum_k \left(a_k\varphi_k + b_k^\dagger\varphi_k^\ast\right).    
    \]
    Define the three operators by their action on the ladder operators $a_k$ and $b_k$:
    \begin{align*}
        P a_k P = a_{\bar{k}}, &\qquad P b_k P = b_{\bar{k}},\\
        T a_k T = a_{\bar{k}}, &\qquad T b_k T = b_{\bar{k}},\\
        C a_k C = b_k, &\qquad C b_k C = a_k,
    \end{align*}
    where $\bar{k} = (\omega_k, -\vec{k})$, and remember $T$ is antiunitary. Thus the actions of
    $C$, $P$ and $T$ on $\phi(x)$ are
    \begin{align*}
        P\phi(t, \vec{x})P & = \sum_k \left(
            a_{\bar{k}}\varphi_k + b_{\bar{k}}^\dagger\varphi_k^\ast    
        \right)(t, \vec{x})
        = \sum_k \left(
            a_k\varphi_k + b_k^\dagger\varphi_k^\ast
        \right)(t, -\vec{x}) = \phi(t, -\vec{x}),\\
        %
        T \phi(t, \vec{x}) T & = \sum_k \left(
            a_{\bar{k}}\varphi_k^\ast + b_{\bar{k}}^\dagger\varphi_k
        \right)(t, \vec{x}) = \sum_k\left(
            a_k\varphi_k + b_k^\dagger\varphi_k^\ast
        \right)(-t, \vec{x}) = \phi(-t, \vec{x}),\\
        %
        C \phi(t, \vec{x}) C & = \sum_k \left(
            b_k\varphi_k + a_k^\dagger\varphi_k^\ast
        \right)(t, \vec{x}) = \phi^\ast(t, \vec{x}).
    \end{align*}
    The transformation properties of $J^\mu$ are
    \begin{align*}
        P J^\mu(t, \vec{x}) P & = (-1)^\mu J^\mu(t, -\vec{x}),\\
        T J^\mu(t, \vec{x}) T & 
        = -i[-(-1)^\mu] (\phi^\ast\partial^\mu\phi - \partial^\mu\phi^\ast\phi)(-t, \vec{x})
        = (-1)^\mu J^\mu(-t, \vec{x}),\\
        C J^\mu(t, \vec{x}) C & 
        = i(\phi\partial^\mu\phi^\ast - \partial^\mu\phi\phi^\ast)
        = -J^\mu(t, \vec{x}),
    \end{align*}
    where $(-1)^\mu \equiv (1, -1, -1, -1)$.

    \item Any Hermitian Lorentz invariant operator must be built from a set of operators that are
    simutaneously Hermitian and Lorentz invariant. The Lorentz invariant operators are:
    \begin{equation*}
        \bar{\psi}\psi, \bar{\psi}\gamma^\mu\psi, \bar{\psi}\sigma^{\mu\nu}\psi,
        i\bar{\psi}\gamma^\mu\gamma^5\psi, \bar{\psi}\gamma^5\psi, |\phi|^2.
    \end{equation*}
    Except $\bar{\psi}\gamma^\mu\gamma^5\psi$ and $\bar{\psi}\gamma^\mu\psi$ have $CPT = -1$, others all has $CPT = +1$. But these two 
    must contract with each other to form a scalar operator, thus guarentees the operator has $CPT = +1$.
\end{problembody}

\problem \textbf{Bound states.} Two spin-$1/2$ particles can combine to a state of total spin either 0 or 1. The 
wave function for these states are odd and even, respectively, under interchange of the two spins.
\begin{problembody}
    \item Use this information to compute the quantum numbers under $P$ and $C$ of all electron-positron
    bound states with $S$, $P$, or $D$ wavefunctions.

    \item Since the electron-photon coupling is given by the Hamitonian
    \begin{equation*}
        \Delta H = \int \, \dd^3x \, e A_\mu j^\mu,
    \end{equation*}
    where $j^\mu$ is the electric current, electrodynamics is invariant to $P$ and $C$ if the components of the vector
    potential have the same $P$ and $C$ parity as the corresponding components of $j^\mu$. Show that this implies the
    following surprising fact: The spin-0 ground state of positronium can decay to 2 photons, but the spin-1 ground state
    must decay to 3 photons. Find the selection rules for the annihilation of higher positronium states, and for 1-photon
    transitions between positronium levels.
\end{problembody}

\solution
\begin{problembody}
    \item The spacial wavefunction of a system with an electron and positron can be written as
    \begin{equation*}
        \Psi(\vec{r}_{e-}, \vec{r}_{e+}) = \psi(\vec{r})\Phi(\vec{R}), 
        \qquad \text{where} \, \vec{r} = \vec{r}_{e-} - \vec{e+},
        \vec{R} = \frac{\vec{e-} + \vec{e+}}{2}.
    \end{equation*}
    $\Phi(\vec{R})$ represents the motion of the centre of mass, $\psi(\vec{r})$ represents the relative motion between
    the electron and the positron, i.e., the internal state of the positronium. The bound state can now be written as
    \begin{equation*}
        \ket{B} = \sqrt{2M_B}\sum_{k, m_1, m_2}\frac{1}{\sqrt{V}}
        \tilde{\psi}_{nlm}(\vec{k})\chi_{ij}^s 
        a_{\vec{k}}^{i\dagger} b_{-\vec{k}}^{j\dagger}
        \ket{0},
    \end{equation*}
    where $M_B$ is the mass of the positronium, and $\tilde{\psi}_{nlm}(\vec{k})$ is the positronium wave function in momentum space:
    \begin{equation*}
        \tilde{\psi}_{nlm}(\vec{k}) = \int \dd^3x \, 
        e^{-i\vec{k}\cdot\vec{r}}
        \psi_{nlm}(\vec{k} / 2)
    \end{equation*}
    and $\chi_{ij}^s$ is the spin wavefunction, satisfies $\chi_{ji}^s = (-1)^{s + 1}\chi_{ij}^s$ according to the information.
    The actions of $P$ and $C$ on $\ket{B}$ are
    \begin{align*}
        P\ket{B} & = -\sum_{k, i, j}\frac{1}{\sqrt{V}}
        \tilde{\psi}_{nlm}(\vec{k})\chi_{ij}^s 
        a_{-\vec{k}}^{i\dagger} b_{\vec{k}}^{j\dagger}
        \ket{0}\\
        & = -\sum_{k, i, j}\frac{1}{\sqrt{V}}
        \tilde{\psi}_{nlm}(-\vec{k})\chi_{ij}^s 
        a_{\vec{k}}^{i\dagger} b_{-\vec{k}}^{j\dagger}
        \ket{0} = (-1)^{l + 1}\ket{B},\\
        %
        C\ket{B} & = \sum_{k, i, j}\frac{1}{\sqrt{V}}
        \tilde{\psi}_{nlm}(\vec{k})\chi_{ij}^s 
        b_{\vec{k}}^{i\dagger} a_{-\vec{k}}^{j\dagger}
        \ket{0}\\
        & = -\sum_{k, i, j}\frac{1}{\sqrt{V}}
        \tilde{\psi}_{nlm}(-\vec{k})\chi_{ji}^s 
        a_{\vec{k}}^{i\dagger} b_{-\vec{k}}^{j\dagger}
        \ket{0} = (-1)^{l + s}\ket{B}.
    \end{align*}

    \item The actions of $P$ and $C$ on $j^\mu$ are
    \begin{equation*}
        Pj^\mu P = (-1)^\mu j^\mu, \qquad Cj^\mu C = -j^\mu,
    \end{equation*}
    so the action on $A_\mu$ is the same
    \begin{equation*}
        PA^\mu P = (-1)^\mu A^\mu, \qquad CA^\mu C = -A^\mu.
    \end{equation*}
    To find the action of $P$ and $C$ on photon states, expand $A^\mu$ in terms of ladder operators as
    \begin{equation*}
        A^\mu = \sum_{k, \sigma} \left(
            a_k^\sigma\epsilon_\sigma^\mu(k)\varphi_k
            + a_k^{\sigma\dagger}\epsilon_\sigma^{\mu\ast}(k)\varphi_k^\ast
        \right),
    \end{equation*}
    where $\epsilon_\sigma^\mu(k)$ are the polarization vectors, $\sigma = 0, 1$ corresponds to scalar 
    and longlitude polarization, they are unphysical states, while $\sigma = 2, 3$ corresponds to 
    two physical polarization states, left handed and right handed, respectively. It also satisfies 
    $\epsilon_\sigma^\mu(-k) = (-1)^\sigma\epsilon_\sigma^\mu$, where $(-k)^\mu = (\omega_k, -\vec{k})$.
    Thus, $a_k^{2\dagger}$ and $a_k^{3\dagger}$ creates left hand and right hand polarized photons 
    respectively, the actions of $P$ and $C$ on them are
    \begin{equation*}
        P a_k^\sigma P = a_{-k}^\sigma, \qquad C a_k^\sigma C = -a_k^\sigma,
    \end{equation*}
    with $\sigma = 2, 3$. That is, a photon has parity $+1$ and charge conjugate $-1$. Any state could be 
    written as a series of creation operators acting on the vacuum $\ket{0}$, so a state with $n$ photons has
    parity $+1$ and charge conjugate $(-1)^n$. The amplitude of a process is given by the \textit{scattering matrix} $S$,
    whose elements are given by
    \begin{equation}\label{equ:cp3:s_matrix_def}
        S_{fi} = \bra{\phi_f}S\ket{\phi_i} = \inner{\psi^+_f}{\psi^-_i},
    \end{equation}
    where $\ket{\psi^\pm_a}$ are the in and out states, they are related to the free states $\ket{\phi_a}$ by
    the Lippmann-Swinger equation
    \begin{equation*}
        \ket{\psi^\pm_a} = \ket{\phi_a} 
        + \frac{1}{E_a - H_0 \pm i\epsilon} V \ket{\psi^\pm_a},
    \end{equation*}
    where $H_0$ is the free Hamitonian, and $V$ is the interaction, i.e., $\Delta H$. Thus, if an operator commutes
    with both $H_0$ and $V$, its action on $\ket{\psi^\pm_a}$ is exactly the same as on the corresponding free state.
    Suppose the action of one of such operators $F$ on $\ket{\phi_a}$ is
    \begin{equation*}
        F\ket{\phi_a} = \eta_a\ket{\phi_{Fa}},
    \end{equation*}
    where $\eta_a$ is some phase, for $F = P$, $\eta_a$ is the product of parities of all particles in that state. Then 
    its action on $\ket{\psi^\pm_a}$ is just the same. Substitute it into \eqref{equ:cp3:s_matrix_def} gives
    \begin{equation}\label{equ:cp3:s_trans}
        S_{fi} = \eta_i \eta_f^\ast S_{Ff, Fi}.
    \end{equation}
    Now apply \eqref{equ:cp3:s_trans} to this problem. Let the final state be $n$ photons with definate momenta and polarizations,
    for $F = P$, the initial state is an eigent state of $F$, with $\eta_i = (-1)^{l + 1}$. While the final state is not 
    always an eigent state of $F$, and $\eta_f = 1$. Thus \eqref{equ:cp3:s_trans} tells us that when $l$ of the positronium is
    even (or odd), the scattering matrix element must be odd (or even) under the inverse of every final state momenta. 

    For the charge conjugate, both the initial state and final state are eigent state of $F$, so in order for $S_{fi}$ not to vanish,
    we must have $\eta_i \eta_f^\ast = (-1)^{l + s + n} = 1$. Thus, the ground spin-0 state can decay into an even number of photons, 
    while spin-1 can decay to an odd number of photons, and the leading decay mode is three photons, other modes are negligible.
\end{problembody}