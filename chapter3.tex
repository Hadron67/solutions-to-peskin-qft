\section{Chapter 3}

\setcounter{equation}{150}
\problem \textbf{Lorentz group.} Recall from Eq. (3.17) the Lorentz commutation relations,
\begin{equation*}
    [J^{\mu\nu}, J^{\rho\sigma}] = i(g^{\nu\rho}J^{\mu\sigma} - g^{\mu\rho}J^{\nu\sigma} - g^{\nu\sigma}J^{\mu\rho} + g^{\mu\sigma}J^{\nu\rho})
\end{equation*}
\begin{enumerate}[label = {(\alph*)}, ref = {(\alph*)}]
    \item \label{ex:cp3:lor_gen_com} 
    Define the generators of rotations and boosts as
    \begin{equation*}
        L^i = \frac{1}{2}\epsilon^{ijk} L^k, \quad K^i = J^{0i}
    \end{equation*}
    where $i, j, k = 1, 2, 3$. An infinitesimal Lorentz transformation can then be written
    \begin{equation*}
        \Phi \to \left(1 - i\vec{\theta}\cdot\vec{L} - i\vec{\beta}\cdot\vec{K}\right)\Phi
    \end{equation*}
    Write the commutation relations of these operators explicitly. (For example, $[L^i, L^j] = i\epsilon^{ijk}L^k$)
    Show that the combinations 
    \begin{equation*}
        \vec{J}_{+} = \frac{1}{2} \left(\vec{L} + i\vec{K}\right) \quad \text{and} 
        \quad \vec{J}_{-} = \frac{1}{2} \left(\vec{L} - i\vec{K}\right)
    \end{equation*} 
    commute with one another and separately satisfy the commutation relations of angular momentum.

    \item The finite-dimensional representations of the rotation group correspond precisely to the allowed values for 
    angular momentum: integers or half-integers. The result of part \ref{ex:cp3:lor_gen_com} implies that all finite-dimensional
    representations of the Lorentz group correspond to pairs of integers or half integers, $(j_+, j_-)$, corresponding to pairs 
    of representations of the rotation group. Using the fact that $\vec{J} = \vec{\sigma} / 2$ in the spin-laws of the 2-component
    objects transforming according to the $(\frac{1}{2}, 0)$ and $(0, \frac{1}{2})$ representations of the Lorentz group. Show 
    that these correspond precisely to the transformations of $\psi_L$ and $\psi_R$ given in (3.27).

    \item The identity $\vec{\sigma}^T = -\sigma^2 \vec{\sigma}\sigma^2$ allows us to rewrite the $\psi_L$ transformation in the 
    unitarily equivalent form
    \begin{equation*}
        \psi^\prime \to \psi^\prime \left(1 + i\vec{\theta}\cdot\frac{\vec{\sigma}}{2} + \vec{\beta}\cdot\frac{\vec{\sigma}}{2}\right),
    \end{equation*}
    where $\psi^\prime = \psi_L^T \sigma^2$. Using this law, we can represent the object that transforms as $(\frac{1}{2}, \frac{1}{2})$
    as a $2 \times 2$ matrix that has the $\psi_R$ transformation law on the left and, simutaneously, the transposed $\psi_L$ transformation
    on the right. Parametrize this matrix as
    \begin{equation*}
        \begin{pmatrix}
            V^0 + V^3  & V^1 - iV^2 \\
            V^1 + iV^2 & V^0 - V^3
        \end{pmatrix}.
    \end{equation*}
    Show that the object $V^\mu$ transformations as a 4-vector.
\end{enumerate}

\solution
\begin{enumerate}[label = {(\alph*)}]
    \item 
\end{enumerate}