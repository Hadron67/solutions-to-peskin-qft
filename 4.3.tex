\problem \textbf{Linear sigma model.} The interactions of pions at low energy can be described
by a phenomenological model called the \textit{linear sigma model}. Essentially, this model 
consists of $N$ real scalar fields coupled by a $\phi^4$ interaction that is symmetric under 
rotations of the $N$ fields. More specifically, let $\Phi^i(x)$, $i = 1, \cdots, N$ be a set of 
$N$ fields, governed by the Hamitonian
\begin{equation*}
    H = \int \dd^3x \, \left(
        \frac{1}{2}(\Pi^i)^2 + \frac{1}{2}(\nabla\Phi^i)^2 + V(\Phi^2)    
    \right),
\end{equation*}
where $(\Phi^i)^2 = \Phi\cdot\Phi$, and
\begin{equation*}
    V(\Phi^2) = \frac{1}{2}m^2(\Phi^i)^2 
    + \frac{\lambda}{4}\left((\Phi^i)^2\right)^2
\end{equation*}
is a function symmetric under rotations of $\Phi$. For (classical) field configurations of $\Phi^i(x)$
that are constant in space and time, this term gives the only contribution to $H$; hence, $V$ is the field
potential energy.

(What does this Hamitonian have to do with the strong interactions? There are two types of light quarks, 
$u$ and $d$. These quarks have identical strong interactions, but different masses. If these quarks are
massless. If these quarks are massless, the Hamitonian of the strong interactions is invariant to unitary
transformations of the 2-component object $(u, d)$:
\begin{equation*}
    \begin{pmatrix}
        u \\ d
    \end{pmatrix} \to \exp(i\vec{\alpha}\cdot\vec{\sigma} / 2)
    \begin{pmatrix}
        u \\ d
    \end{pmatrix}.
\end{equation*}
This is transformation is called an \textit{isospin} rotation. If, in addition, the strong interactions 
are described by a vector ``gluon'' field (as is true in QCD), the strong interaction Hamitonian is invariant
to the isospin rotations done separately on the left-handed and right handed components of the quark fields.
Thus, the complete symmetry of QCD with two massless quarks is $SU(2)\times SU(2)$. It happens that $SO(4)$,
the group of rotations in 4 dimensions, is isomorphic to $SU(2)\times SU(2)$
\footnote{I wonder if this is a mistake, or expressed wrongly, since $SO(4)$ is not a direct product of Lie groups.}, 
so for $N = 4$, the linear 
sigma model has the same symmetry group as the strong interactions.)
\begin{problembody}
    \item Analyze the linear sigma model for $m^2 > 0$ by noticing that, for $\lambda = 0$, the Hamitonian
    given above is exactly $N$ copies of the Klein-Gordon Hamitonian. We can the calculate scattering amplitudes
    as perturbation series in the parameter $\lambda$. Show that the propagator is
    \begin{equation*}
        \wick{\c \Phi^i(x) \c \Phi^j(y)} = \delta^{ij}D_F(x - y),
    \end{equation*}
    where $D_F$ is the standard Klein-Gordon propagator for mass $m$, and that there is one type of vertex gien 
    by
    \begin{equation*}
        \begin{tikzpicture}[baseline = (o.south)]
            \node (k) at (-1,  1) {$k$};
            \node (l) at ( 1,  1) {$l$};
            \node (i) at (-1, -1) {$i$};
            \node (j) at ( 1, -1) {$j$};
            \draw (k) -- (0, 0) -- (l);
            \draw (i) -- (0, 0) -- (j);
            \node (o) [vertex] at (0, 0) {};
        \end{tikzpicture}
        = -2i\lambda(
            \delta^{ij}\delta^{kl}
            + \delta^{il}\delta^{jk}
            + \delta^{ik}\delta^{jl}
        ).
    \end{equation*}
    (That is, the vertex between two $\Phi^1$s and two $\Phi^2$s has the value $(-2i\lambda)$; that between four 
    $\Phi^1$s has the value $(-6i\lambda)$.) Compute, to leading order in $\lambda$, the differential cross sections 
    $\dd\sigma / \dd\Omega$, in the center of mass frame, for the scattering processes
    \begin{equation*}
        \Phi^1\Phi^2 \to \Phi^1\Phi^2, 
        \qquad \Phi^1\Phi^1 \to \Phi^2\Phi^2,
        \qquad \text{and} \qquad
        \Phi^1\Phi^1 \to \Phi^1\Phi^1
    \end{equation*}
    as functions of the center-of-mass energy.

    \item Now consider the case $m^2 < 0$: $m^2 = -\mu^2$. In this case, $V$ has a local maximum, rather than a minimum,
    at $\Phi^i = 0$. Since $V$ is a potential energy, this implies that the ground state of the theory is not near $\Phi^i = 0$
    but rather is obtained by shifting $\Phi^i$ toward the minimum of $V$. By rotational invariance, we can consider this 
    shift to be in the $N$th direction. Write, then,
    \begin{align*}
        \Phi^i(x) & = \pi^i(x), \qquad i = 1, \cdots, N - 1,\\
        \Phi^N(x) & = v + \sigma(x),
    \end{align*}
    where $v$ is a constant chosen to minimize $V$. (The notation $\pi^i$ suggests a poin field and should not be confused
    with a canonical momentum.) Show that, in these new coordinates (and substituting for $v$ its expression in terms of $\lambda$
    and $\mu$), we have a theory of a massive $\sigma$ field and $N - 1$ \textit{massless} pion fields, interacting through 
    cubic and quartic potential energy terms which all become small as $\lambda \to 0$. Construct the Feynman rules by assigning
    values to the propagators and vertices:
    \begin{align*}
        \wick{\c \sigma \c \sigma} & = \qquad
        \begin{tikzpicture}[baseline = (o.south)]
            \node (o) [empty]  at (0, 0) {};
            \draw [double line] (0, 0) -- (1.5, 0) [arrowed];
        \end{tikzpicture}
        \qquad\qquad
        \begin{tikzpicture}[baseline = (o.south), scale = 0.6]
            \node (i) at (-150:1.3) {$i$};
            \node (j) at ( -30:1.3) {$j$};
            \draw (-150:1) -- (0, 0) -- (-30:1);
            \draw [double line] (0, 1) -- (0, 0);
            \node (o) [vertex] at (0, 0) {};
        \end{tikzpicture}
        \qquad
        \begin{tikzpicture}[baseline = (o.south), scale = 0.6]
            \node (k) [empty] at (0, 1) {};
            \node (i) [empty] at (-150:1) {};
            \node (j) [empty] at (-30:1) {};
            \draw [double line] (i) -- (0, 0) -- (j);
            \draw [double line] (k) -- (0, 0);
            \node (o) [vertex] at (0, 0) {};
        \end{tikzpicture}\\
        %
        \wick{\c \pi^i \c \pi^j} & = \qquad
        \begin{tikzpicture}[baseline = (i.south)]
            \node (j) [empty, label = right:$j$] at (1.5, 0) {};
            \node (i) [empty, label = left:$i$] at (0, 0) {};
            \draw (0, 0) -- (1.5, 0) [arrowed];
        \end{tikzpicture}
        \qquad\qquad
        \begin{tikzpicture}[baseline = (o.south), scale = 0.6]
            \node (o) [vertex] at (0, 0) {};
            \draw (135:1) -- (0, 0) -- (45:1);
            \draw (-135:1) -- (0, 0) -- (-45:1);
            \node at (135:1.2) {$k$};
            \node at (45:1.2) {$l$};
            \node at (-135:1.2) {$i$};
            \node at (-45:1.2) {$j$};
        \end{tikzpicture}
        \qquad
        \begin{tikzpicture}[baseline = (o.south), scale = 0.6]
            \draw [double line] (135:1) -- (0, 0) -- (45:1);
            \node (o) [vertex] at (0, 0) {};
            \draw (-135:1) -- (0, 0) -- (-45:1);
            \node at (-135:1.2) {$i$};
            \node at (-45:1.2) {$j$};
        \end{tikzpicture}
        \qquad
        \begin{tikzpicture}[baseline = (o.south), scale = 0.6]
            \draw [double line] (135:1) -- (0, 0) -- (45:1);
            \draw [double line] (-135:1) -- (0, 0) -- (-45:1);
            \node (o) [vertex] at (0, 0) {};
        \end{tikzpicture} 
    \end{align*}

    \item Compute the scattering amplitude for the process
    \begin{equation*}
        \pi^i(p_1)\pi^j(p_2) \to \pi^k(p_3)\pi^l(p_4)
    \end{equation*}
    to leading order in $\lambda$. There are now four Feynman diagrams that contribute:
    \begin{equation*}
        \begin{tikzpicture}[baseline = (o.south), scale = 0.8]
            \coordinate (a) at (0, 0.6);
            \coordinate (b) at (0, -0.6);
            \node (o) [empty] at (0, 0) {};
            \draw [double line] (b) -- (a);
            \draw (a) + (150:1) -- (a) -- +(30:1);
            \draw (b) + (-150:1) -- (b) -- +(-30:1);
        \end{tikzpicture}
        \qquad + \qquad
        \begin{tikzpicture}[baseline = (o.south), scale = 0.8]
            \coordinate (a) at (0.6, 0);
            \coordinate (b) at (-0.6, 0);
            \node (o) [empty] at (0, 0) {};
            \draw [double line] (b) -- (a);
            \draw (a) + (60:1) -- (a) -- +(-60:1);
            \draw (b) + (120:1) -- (b) -- +(-120:1);
        \end{tikzpicture}
        \qquad + \qquad
        \begin{tikzpicture}[baseline = (o.south), scale = 0.8]
            \coordinate (a) at (0.6, 0);
            \coordinate (b) at (-0.6, 0);
            \node (o) [empty] at (0, 0) {};
            \draw [double line] (b) -- (a);
            \draw (b) + (120:1) -- (a) -- +(-60:1);
            \draw (a) + (60:1) -- (b) -- +(-120:1);
        \end{tikzpicture}
        \qquad + \qquad
        \begin{tikzpicture}[baseline = (o.south), scale = 0.8]
            \node (o) [empty] at (0, 0) {};
            \draw (-1, 1) -- (1, -1);
            \draw (-1, -1) -- (1, 1);
        \end{tikzpicture}
    \end{equation*}
    Show that, at threshold ($\vec{p}_i = 0$), these diagrams sum to \textit{zero}. (Hint: It may be easier to first consider
    the specific process $\pi^1\pi^1 \to \pi^2\pi^2$, for which only the first and fourth diagrams are nonzero, before tackling 
    the general case.) Show that, in the special case $N = 2$ (1 species of pion), the term of $\bigo(p^2)$ also cancels.

    \item Add to $V$ a symmetriy-breaking term,
    \begin{equation*}
        \Delta V = -a\Phi^N,
    \end{equation*}
    where $a$ is a (small) constant. (In QCD, a term of this form is produces if the $u$ and $d$ quarks have the same nonvanishing
    mass.) Find the new value of $v$ that minimizes $V$, and work out the content of the theory about that point. Show that the
    pion acquires a mass such that $m_\pi^2 \sim a$, and show that the pion scattering amplitude at threshold is now nonvanishing
    and also proportional to $a$.
\end{problembody}

\solution
\begin{problembody}
    \item The propagator for this Hamitonian is obvious. Vertex rule can be obtained by considering
    \begin{equation}\label{equ:cp3:vertex_contract}
        \bra{0}T \left[ \Phi^k(x_k)\Phi^l(x_l)\Phi^i(x_i)\Phi^j(x_j) 
            \int \dd^4x \, \frac{-i\lambda}{4} 
            \sum_{m = 1}^N \sum_{n = 1}^N (\Phi^m)^2(\Phi^n)^2
        \right]\ket{0},
    \end{equation}
    and only consider connected part, i.e. the four $\Phi^i$'s outside the integration shouldn't contract with each other. The interchange
    of the $\Phi^m$ and $\Phi^n$ gives a factor of two, while the contraction of the four $\Phi^i$'s with $\Phi^m$ and $\Phi^n$ gives 2, 
    respectively. Thus \eqref{equ:cp3:vertex_contract} becomes
    \begin{equation*}
        -2i\lambda \left(
            \delta^{ij}\delta^{kl} 
            + \delta^{ik}\delta^{jl}
            + \delta^{il}\delta^{jk}    
        \right)
        \int \dd^4x \, D_F(x_k - x)D_F(x_l - x)D_F(x_i - x)D_F(x_j - x).
    \end{equation*}
    So that the vertex rule is given by
    \begin{equation*}
        \begin{tikzpicture}[baseline = (o.south)]
            \node (k) at (-1,  1) {$k$};
            \node (l) at ( 1,  1) {$l$};
            \node (i) at (-1, -1) {$i$};
            \node (j) at ( 1, -1) {$j$};
            \draw (k) -- (0, 0) -- (l);
            \draw (i) -- (0, 0) -- (j);
            \node (o) [vertex] at (0, 0) {};
        \end{tikzpicture}
        = -2i\lambda(
            \delta^{ij}\delta^{kl}
            + \delta^{il}\delta^{jk}
            + \delta^{ik}\delta^{jl}
        ).
    \end{equation*}
    Thus
    \begin{equation*}
        i\ims(\Phi^i\Phi^j \to \Phi^k\Phi^l) = -2i\lambda \left(
            \delta^{ij}\delta^{kl}
            + \delta^{il}\delta^{jk}
            + \delta^{ik}\delta^{jl}
        \right) \equiv -2i\lambda A,
    \end{equation*}
    and the cross section
    \begin{equation*}
        \df{\sigma}{\Omega}(\Phi^i\Phi^j \to \Phi^k\Phi^l)
        = \frac{\abs{\ims}^2}{64\pi^2 E_{CM}^2}
        = \frac{\lambda^2 A^2}{16 \pi^2 E_{CM}^2}.
    \end{equation*}
    For the three processes in concern, we have $A = 1, 1, 3$ respectively. As we can see, all of them are isotropic.

    \item In this case, $V$ is given by
    \begin{equation*}
        V = -\frac{1}{2}\mu^2 (\Phi^i)^2 + \frac{\lambda}{4}\left((\Phi^i)^2\right)^2,
    \end{equation*}
    which is quadratic in $(\Phi^i)^2$, and has minimum at $(\Phi^i)^2 = \mu^2 / \lambda$. Thus $v = \mu / \sqrt{\lambda}$. Then
    $V$ becomes
    \begin{align*}
        V & = \frac{\lambda}{4} \left[ (\pi^i)^2 + (v + \sigma)^2\right]^2
        - \frac{1}{2}\mu^2 \left[ (\pi^i)^2 + (v + \sigma)^2 \right]\\
        %
        & = \frac{\lambda}{4}\left((\pi^i)^2\right)^2 + \frac{\lambda}{4}\sigma^4 
        + \sqrt{\lambda} \mu (\pi^i)^2\sigma 
        + \frac{\lambda}{2}(\pi^i)^2\sigma^2
        + \sqrt{\lambda}\mu \sigma^3
        + \mu^2\sigma^2
        - \frac{\mu^4}{4\lambda}.
    \end{align*}
    There is no $(\pi^i)^2$ term, so the $\pi^i$'s are massless, while the term $\mu^2\sigma^2$ indicates $\sigma$ field has mass $\sqrt{2}\mu$, thus the 
    propagators are
    \begin{equation*}
        \wick{\c \pi^i \c \pi^j} = \frac{i\delta^{ij}}{k^2 + i\epsilon}, \qquad
        \wick{\c \sigma \c \sigma} = \frac{i}{k^2 - 2\mu^2 + i\epsilon}.
    \end{equation*}
    Feynman rule can be read directly from $V$:
    \begin{equation*}
        \begin{tikzpicture}[baseline = (o.south), scale = 0.6]
            \draw (135:1) -- (0, 0) -- (45:1);
            \draw (-135:1) -- (0, 0) -- (-45:1);
            \node at (135:1.3) {$k$};
            \node at (45:1.3) {$l$};
            \node at (-135:1.3) {$i$};
            \node at (-45:1.3) {$j$};
            \node (o) [vertex] at (0, 0) {};
        \end{tikzpicture}
        = -2i\lambda\left(
            \delta^{kl}\delta^{ij}
            + \delta^{ki}\delta^{lj}
            + \delta^{kj}\delta^{li}    
        \right),
        \qquad
        \begin{tikzpicture}[baseline = (o.south), scale = 0.6]
            \draw [double line] (135:1) -- (0, 0) -- (45:1);
            \draw [double line] (-135:1) -- (0, 0) -- (-45:1);
            \node (o) [vertex] at (0, 0) {};
        \end{tikzpicture} = -6i\lambda,
    \end{equation*}
    \begin{equation*}
        \begin{tikzpicture}[baseline = (o.south), scale = 0.6]
            \draw [double line] (0, 0) -- (0, 1);
            \draw (0, 0) -- (-150:1);
            \draw (0, 0) -- (-30:1);
            \node at (-150:1.3) {$i$};
            \node at (-30:1.3) {$j$};
            \node (o) [vertex] at (0, 0){};
        \end{tikzpicture} = -2i\sqrt{\lambda}\mu \delta^{ij},
        \qquad
        \begin{tikzpicture}[baseline = (o.south), scale = 0.6]
            \draw [double line] (0, 0) -- (0, 1);
            \draw [double line] (-30:1) -- (0, 0) -- (-150:1);
            \node (o) [vertex] at (0, 0){};
        \end{tikzpicture} = -6i\sqrt{\lambda}\mu,
        \qquad
        \begin{tikzpicture}[baseline = (o.south), scale = 0.6]
            \draw [double line] (135:1) -- (0, 0) -- (45:1);
            \draw (-135:1) -- (0, 0) -- (-45:1);
            \node at (-135:1.3) {$i$};
            \node at (-45:1.3) {$j$};
            \node (o) [vertex] at (0, 0) {};
        \end{tikzpicture} = -2i\lambda\delta^{ij}.
    \end{equation*}

    \item We evaluate each diagram:
    \begin{align*}
        \begin{tikzpicture}[baseline = (o.south)]
            \coordinate (a) at (0, 0.6);
            \coordinate (b) at (0, -0.6);
            \node (o) [empty] at (0, 0) {};
            \draw [double line] (b) -- (a);
            \draw (a) -- +(150:1) [arrowed, text = $p_3$];
            \draw (a) -- +(30:1) [arrowed, texti = $p_4$];
            \node at ($ (a) + (150:1.3) $) {$k$};
            \node at ($ (a) + (30:1.3) $) {$l$};
            \draw ($ (b) + (-150:1) $) -- (b) [arrowed, text = $p_1$];
            \draw ($ (b) + (-30:1) $) -- (b) [arrowed, texti = $p_2$];
            \node at ($ (b) + (-150:1.3) $) {$i$};
            \node at ($ (b) + (-30:1.3) $) {$j$};
        \end{tikzpicture}
        & = -4\lambda\mu^2\delta^{ij}\delta^{kl}\frac{i}{(p_1 + p_2)^2 - 2\mu^2},\\
        %
        \begin{tikzpicture}[baseline = (o.south)]
            \coordinate (a) at (-0.6, 0);
            \coordinate (b) at (0.6, 0);
            \node (o) [empty] at (0, 0) {};
            \draw [double line] (b) -- (a);
            \draw (a) -- +(120:1) [arrowed, texti = $p_3$];
            \draw (b) -- +(60:1) [arrowed, text = $p_4$];
            \node at ($ (a) + (120:1.3) $) {$k$};
            \node at ($ (b) + (60:1.3) $) {$l$};
            \draw ($ (a) + (-120:1) $) -- (a) [arrowed, texti = $p_1$];
            \draw ($ (b) + (-60:1) $) -- (b) [arrowed, text = $p_2$];
            \node at ($ (a) + (-120:1.3) $) {$i$};
            \node at ($ (b) + (-60:1.3) $) {$j$};
        \end{tikzpicture}
        & = -4\lambda\mu^2\delta^{ki}\delta^{lj}\frac{i}{(p_3 - p_1)^2 - 2\mu^2},\\
        \begin{tikzpicture}[baseline = (o.south)]
            \coordinate (a) at (-0.6, 0);
            \coordinate (b) at (0.6, 0);
            \node (o) [empty] at (0, 0) {};
            \draw [double line] (a) -- (b);
            \draw (a) -- node[midway, label=above:$p_4$] {} ($ (b) + (60:1) $) [arrowed];
            \draw (b) -- node[midway, label=above:$p_3$] {} ($ (a) + (120:1) $) [arrowed];
            \node at ($ (a) + (120:1.3) $) {$k$};
            \node at ($ (b) + (60:1.3) $) {$l$};
            \draw ($ (a) + (-120:1) $) -- (a) [arrowed, texti = $p_1$];
            \draw ($ (b) + (-60:1) $) -- (b) [arrowed, text = $p_2$];
            \node at ($ (a) + (-120:1.3) $) {$i$};
            \node at ($ (b) + (-60:1.3) $) {$j$}; 
        \end{tikzpicture}
        & = -4\lambda\mu^2\delta^{li}\delta^{kj}\frac{i}{(p_4 - p_1)^2 - 2\mu^2},\\
        \begin{tikzpicture}[baseline = (o.south)]
            \draw (135:1) -- (0, 0) -- (-45:1);
            \draw (45:1) -- (0, 0) -- (-135:1);
            \node (o) [empty] at (0, 0) {};
            \node at (135:1.3) {$k$};
            \node at (45:1.3) {$l$};
            \node at (-45:1.3) {$j$};
            \node at (-135:1.3) {$i$};
        \end{tikzpicture}
        & = -2i\lambda\left(
            \delta^{kl}\delta^{ij}
            + \delta^{ki}\delta^{lj}
            + \delta^{kj}\delta^{li}    
        \right).
    \end{align*}
    The amplitude is the sum of the above diagrams:
    \begin{equation*}
        i\ims = -2i\lambda\delta^{ij}\delta^{kl}\frac{s}{s - 2\mu^2}
        - 2i\lambda\delta^{ik}\delta^{lj}\frac{t}{t - 2\mu^2}
        - 2i\lambda\delta^{il}\delta^{kj}\frac{u}{u - 2\mu^2},
    \end{equation*}
    where 
    \begin{equation*}
        s = (p_1 + p_2)^2, \qquad t = (p_3 - p_1)^2, \qquad u = (p_4 - p_1)^2.
    \end{equation*}
    At threshold where $\vec{p}_i = 0$, we have $s = t = u = 0$, so $\ims$ vanishes. In the special case $N = 2$, all
    the $\delta$'s in $\ims$ become factors of one:
    \begin{equation*}
        \ims = -2\lambda\left(
            \frac{s}{s - 2\mu^2}
            + \frac{t}{t - 2\mu^2}
            + \frac{u}{u - 2\mu^2}    
        \right).
    \end{equation*}

    \item With the presence of $\Delta V$, $V$ is now
    \begin{equation*}
        V = \frac{\lambda}{4} \left[
            (\pi^i)^2 + (\Phi^N)^2
        \right]^2
        - \frac{1}{2}\mu^2\left[
            (\pi^i)^2 + (\Phi^N)^2    
        \right]
        - a \Phi^N.
    \end{equation*}
    Let $v$ be the value of $\Phi^N$ that minimizes $V$ when $\pi^i = 0$. The condition is
    \begin{equation*}
        \pd{V}{v} = \lambda v^3 - \mu^2 v - a = 0.
    \end{equation*}
    Since $a$ is small, $v$ can be solved by perturbation. Let $v = \mu / \sqrt{\lambda} + xa + o(a)$, with $x$ be a constant to be
    determined. Substitude into the equation above, we get $x = 1 / 2\mu^2$. The new potential is then
    \begin{align*}
        V & = \frac{\lambda}{4}\left[
            (\pi^i)^2 + (v + \sigma)^2    
        \right] - \frac{1}{2}\mu^2 \left[
            (\pi^i)^2 + (v + \sigma)^2
        \right] - a (v + \sigma)\\
        %
        & = \frac{a\sqrt{\lambda}}{2\mu} (\pi^i)^2 
        + \left(\mu^2 + \frac{3a\sqrt{\lambda}}{2\mu}\right)\sigma^2
        + \left(\frac{a\lambda}{2\mu^2} + \sqrt{\lambda}\mu\right)(\pi^i)^2\sigma
        + \left(\frac{a\lambda}{2\mu^2} + \sqrt{\lambda}\mu\right) \sigma^3\\
        & \qquad + \frac{\lambda}{4}\left((\pi^i)^2\right)^2
        + \frac{\lambda}{2}(\pi^i)^2\sigma^2 + \frac{\lambda}{4}\sigma^4
        -\frac{\mu^4}{4\lambda} - \frac{a\mu}{\sqrt{\lambda}} - o(a).
    \end{align*}
    The $\Delta V$ modifies the particle masses, the $\pi^2\sigma$ and $\sigma^3$ vertices. The pion acquires mass $m_\pi^2 = a\sqrt{\lambda} / \mu \propto a$.
    Since the $\pi^4$ vertex is unaffected, while the coeffecient of $\pi^2\sigma$ vertex is added by $a\lambda / 2\mu^2$, the pion scattering amplitude at threshold 
    now no longer vanish:
    \begin{align*}
        i\ims & = -4i\lambda\left(\mu^2 + \frac{\sqrt{\lambda}}{\mu}a\right)\left(
            \frac{\delta^{ij}\delta^{kl}}{s - m_\sigma^2} 
            + \frac{\delta^{ik}\delta^{lj}}{t - m_\sigma^2} 
            + \frac{\delta^{il}\delta^{kj}}{u - m_\sigma^2}
        \right)\\
        & \qquad -2i\lambda\left(
            \delta^{kl}\delta^{ij}
            + \delta^{ki}\delta^{lj}
            + \delta^{kj}\delta^{li}    
        \right)\\
        %
        & = i\ims_0 + 4i\frac{\lambda\sqrt{\lambda}}{\mu}a \left(
            \frac{\delta^{ij}\delta^{kl}}{s - m_\sigma^2} 
            + \frac{\delta^{ik}\delta^{lj}}{t - m_\sigma^2} 
            + \frac{\delta^{il}\delta^{kj}}{u - m_\sigma^2}
        \right),
    \end{align*}
    where $m_\sigma$ is the mass of ``sigmaon'' with the presence of $\Delta V$, and $\ims$ is the original scattering amplitude with the sigmaon mass in the propagator 
    replaced by $m_\sigma$. Thus $\ims$ vanishes at threshold and the remaining terms are proportional to $a$.
\end{problembody}